%\makeatletter
%\def\toclevel@chapter{0}
%\makeatother


\chapter{Approche spectrale}
%\begin{tikzpicture}[remember picture, overlay]
%\node[anchor=north east,inner sep=0pt] at (current page.north east) {\includegraphics[scale=1]{Fig/Chapter1/g825.png}};
%\end{tikzpicture}
\label{ch:TauS_NJP}

Dans le chapitre précédent, nous nous sommes concentrés sur la mesure du temps de diffusion élastique, que nous avons comparé à la théorie perturbative de Born. En particulier, nous avons montré que celle-ci fournit une prédiction quantitative du temps de diffusion élastique dans le régime de diffusion faible, mais que de fortes déviations apparaissent lorsque l'amplitude du désordre $|\VR|$ augmente et que l'impulsion initiale $k_{\mathrm{i}}$ diminue. La comparaison de nos données à la prédiction de Born nous a de plus permit d'étudier quantitativement la transition entre les régimes de diffusion faible et diffusion forte, révélant ainsi la forte influence de la distribution du potentiel sur la position de cette transition \citep{richard2019elastic}. 

Néanmoins, l'approximation de Born échoue à décrire le régime de diffusion forte. Dans ce chapitre, nous décrirons le comportement du temps de diffusion élastique à l'aide des fonctions spectrales mentionnées au chapitre \ref{ch:Localisation}. Plus particulièrement, nous estimerons celles-ci à l'aide du développement de Born et à l'aide d'une approche auto-consistante. Enfin, nous comparerons nos valeurs expérimentales du temps de diffusion élastique à des valeurs extraites de la mesure expérimentales des fonctions spectrales.

L'étude présentée ici a fait l'objet d'une publication dans la revue \emph{New Journal of Physics} \citep{signoles2019ultracold}.

\section{Temps de diffusion élastique et fonctions spectrales}

Dans un premier temps, nous allons présenter le lien entre le temps de diffusion élastique $\taus(\VR,\mathbf{k}_{\mathrm{i}})$ et la fonction spectrale $A(\mathbf{k}_{\mathrm{i}},E)$. Nous décrirons ensuite les approximations couramment utilisées pour la détermination de cette dernière, dont nous allons extraire un temps de vie $\taus^{\mathrm{sf}}$ que nous confronterons à nos mesures. 


\subsection{Généralités sur la fonction spectrale}
Le hamiltonien $\hat{H}$ d'une particule quantique soumise à un potentiel extérieur peut être décomposé sous la forme
\begin{equation}
\hat{H}=\hat{H}_0+\hat{V} \text{ ,}
\end{equation}
avec $\hat{H}_0$ le hamiltonien de la particule libre et $\hat{V}$ le potentiel auquel est soumis cette particule. 

La fonction spectrale est définie à l'aide de la fonction de Green moyenne par
\begin{equation}
A(\mathbf{k}_{\mathrm{i}},E)= -\frac{1}{\pi} \mathrm{Im}[\overline{G}(E,\mathbf{k}_{\mathrm{i}})]
\end{equation}

Fonction de Green moyenne ($\overline{\cdots}$ représente la moyenne sur les différentes réalisation du désordre):
\begin{equation}
\overline{G}(E,\mathbf{k}_{\mathrm{i}})=\frac{1}{E-E_{\mathrm{k}_i}-\overline{V}-\Sigma(E,\mathbf{k}_{\mathrm{i}})}
\end{equation}
avec $\Sigma(E,\mathbf{k}_{\mathrm{i}})$ la \emph{Self-Energy}.

Fonction de Green de la particule libre
\begin{equation}
G_0(E,\mathbf{k}_{\mathrm{i}})=\frac{1}{E-E_{\mathrm{k}_i}+i0^+}
\end{equation}

Donc in peut écrire la fonction spectrale à l'aide de la self-energy sous la forme
\begin{equation}
A(\mathbf{k}_{\mathrm{i}},E)=-\frac{1}{\pi}\frac{\mathrm{Im}[\Sigma(E,\mathbf{k}_{\mathrm{i}})]}{(E-E_{\mathrm{k},i}-\overline{V}-\mathrm{Re}[\Sigma(E,\mathbf{k}_{\mathrm{i}})])^2+\mathrm{Im}[\Sigma(E,\mathbf{k}_{\mathrm{i}})]^2}
\end{equation}

On montre que l'opérateur évolution moyen de l'état initial est simplement donné par la transformée de Fourier de la fonction spectrale
\begin{equation}
\overline{U}_{\mathrm{k}_{\mathrm{i}}}(t)=\left\langle \mathbf{k}_{\mathrm{i}} | \overline{\exp(-itH)}  | \mathbf{k}_{\mathrm{i}} \right\rangle = \int{\diff E e^{-iEt} A(\mathbf{k}_{\mathrm{i}},E)}
\end{equation}
comme on pouvait s'en douter lorsque le désordre est allumé brusquement: la fonction spectrale correspond alors à la distribution d'énergie de l'état initial dans le désordre. La relation par transformée de Fourier est alors naturelle.

Dans la limite de désordres très forts $|\VR|\gg\ER$, la limite classique, la fonction spectrale converge vers la distribution du potentiel déplacée par l'énergie cinétique initiale de l'onde
\begin{equation}
A(\mathbf{k}_{\mathrm{i}},E)\approx \mathcal{P}(E-E_{\mathrm{k}_i})
\end{equation}


\subsection{Approximation de Born: premier ordre}
\begin{equation}
\frac{\hb}{\taus^{\mathrm{sf}}}=-2 \mathrm{Im}[\Sigma^{(1)}(E_{\mathrm{k}_i},\mathbf{k}_{\mathrm{i}})]=2\pi \sum_{\mathbf{k}'}{\widetilde{C}(\mathbf{k}'-\mathbf{k}_{\mathrm{i}}) \delta(E_{\mathrm{k}'}-E_{\mathrm{k}_i})}
\end{equation}
\subsection{Approximation de Born: second ordre}
\subsection{Approximation de Born auto-consistante}
\begin{equation}
\Sigma_{\mathrm{scba}}(E,\mathbf{k}_{\mathrm{i}})=\widetilde{C}(\mathbf{k}_{\mathrm{i}})\ast \overline{G}_{\mathrm{scba}}(E,\mathbf{k}_{\mathrm{i}})
\end{equation}
\begin{equation}
\overline{G}_{\mathrm{scba}}(E,\mathbf{k}_{\mathrm{i}}) = \frac{1}{E-E_{\mathrm{k}_i} - \overline{V} - \Sigma_{\mathrm{scba}}(E,\mathbf{k}_{\mathrm{i}})}
\end{equation}

\section{Temps de diffusion élastique et fonctions spectrales mesurées pour un désordre de type speckle}
\subsection{Limite de l'approche auto-consistante pour un désordre de type speckle}
\subsection{Mesure des fonctions spectrales}
\subsection{Comparaison du temps de diffusion élastique avec les fonctions spectrales mesurées}
