\makeatletter
\def\toclevel@chapter{0}
\makeatother


\chapter{Approche spectrale}
%\begin{tikzpicture}[remember picture, overlay]
%\node[anchor=north east,inner sep=0pt] at (current page.north east) {\includegraphics[scale=1]{Fig/Chapter1/g825.png}};
%\end{tikzpicture}


Approche fonctions de green, ordre 1, ordre 2, SCBA...
départ quadratique?
mesure fonctions spectrales?

reprendre le NJP, et tenter une explication du départ quadratique. Faire gaffe avec les décroissances!

%\addcontentsline{toc}{subsection}{Départ quadratique à caler quelque part dans ce chapitre}

\section{Temps de diffusion élastique et fonctions spectrales}
\subsection{Généralités sur la fonction spectrale}
\subsection{Approximation de Born: premier ordre}
\subsection{Approximation de Born: second ordre}
\subsection{Approximation de Born auto-consistante}

\section{Temps de diffusion élastique et fonctions spectrales mesurées pour un désordre de type speckle}
\subsection{Limite de l'approche auto-consistante pour un désordre de type speckle}
\subsection{Mesure des fonctions spectrales}
\subsection{Comparaison du temps de diffusion élastique avec les fonctions spectrales mesurées}
