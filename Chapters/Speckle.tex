\chapter{Propriétés d'un désordre de type speckle}
%\begin{tikzpicture}[remember picture, overlay]
%\node[anchor=north east,inner sep=0pt] at (current page.north east) {\includegraphics[scale=1]{Fig/Chapter1/g825.png}};
%\end{tikzpicture}


propriétés speckle "simple", proriétés statistiques, spatiales 3D...
meilleure modélisation avec effets non paraxiaux?

propriétés double speckle (stage M2? - ou alors à compléter rapidement)

\section{Propriétés statistiques d'un champ de speckle}
\subsection{Propriétés du diffuseur}
\subsection{Statistiques de l'intensité d'un speckle}

\section{Corrélations spatiales d'un champ de speckle}
\subsection{Corrélation transverse}
\subsection{Corrélation longitudinale}

\section{Propriétés du potentiel de type speckle}
\subsection{Propriétés du potentiel}
\subsection{Possibilité d'un potentiel dépendant de l'état interne}

\section{Potentiel composé d'un speckle bichromatique}
\subsection{S'éloigner de résonance}
\subsection{Étude de la similitude de deux speckles}