\chapter{Mises à jour de l'expérience}
\label{ch:new_exp}
%\begin{tikzpicture}[remember picture, overlay]
%\node[anchor=north east,inner sep=0pt] at (current page.north east) {\includegraphics[scale=1]{Fig/Chapter1/g825.png}};
%\end{tikzpicture}


parler des modifications apportées à la manip: cicero, ODT + evap, calibration levitation par oscillations + spin-flip, réparation lévitation...

\section{Mise à jour de l'informatique de l'expérience}
Intro sur la part importante de l'informatique sur l'expérience, autant pour le contrôle que pour l'acquisition des images et le traitement de données. 
\subsection{Contrôle de l'expérience: passage à la suite Cicero}
\label{sc:cicero}
Principales fonctionnalités de cicero, présentation du hardware. Limitations ressenties par rapport à l'approche programmatique de Matlab. Développement du serveur d'enregistrement des données pour utilisation dans matlab. Départ à la retraite d'André Villing. 

\subsection{Développement d'une nouvelle interface d'acquisition et de traitement d'images}
Très rapide, peu de détails. Cahier des charges, principales fonctionnalités avec la programation des DDS. 





\section{Réparation et recalibration de la lévitation magnétique}
\subsection{Réparation de la lévitation magnétique}
\label{sc:levitation}
\subsection{Calibration par oscillations}
\subsection{Calibration par radio-fréquences}

\section{Changement du laser telecom et calibration du piège optique}
\subsection{Changement du laser telecom}
\subsection{Calibration du piège optique}
%\subsection{Optimisation de l'évaporation dans le piège dipolaire}

\section{Optimisation de l'évaporation tout-optique}
\label{sc:evap_optique}

