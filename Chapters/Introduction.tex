%\makeatletter
%\def\toclevel@chapter{-1}
%\makeatother

\chapter*{Introduction}
\addcontentsline{toc}{chapter}{Introduction}

Intro générale sur la physique, le contexte et présentation du plan.


\section*{Déroulement de la thèse}
\addcontentsline{toc}{section}{Déroulement de la thèse}


\section*{Structure du manuscrit}
\addcontentsline{toc}{section}{Structure du manuscrit}
Ce manuscrit se décompose selon les six chapitres suivants:
\begin{itemize}
\item[\textendash] \textbf{Chapitre 1:} Nous commencerons par présenter les grandes lignes du transport quantique en milieu désordonné pour introduire le phénomène de localisation d'Anderson, pour ensuite nous attarder sur la transition de phase d'Anderson entre états localisés et états diffusifs. Nous nous concentrerons particulièrement sur l'état de l'art de la transition d'Anderson à l'aide des expériences d'atomes ultra-froids.  Nous ferons ainsi apparaître la quantité centrale de l'étude de la physique du désordre, la fonction spectrale. \\

\item[\textendash] \textbf{Chapitre 2:} Dans un second temps, nous présenterons les grandes lignes de notre plateforme expérimentale. Nous rappellerons donc les principales propriétés d'un condensat de Bose-Einstein, ainsi que les différents processus d'interaction lumière-matière permettant de manipuler des atomes. Enfin, nous présenterons les différentes étapes d'un cycle expérimentales nous permettant d'obtenir un gaz quantique dégénéré. \\

\item[\textendash] \textbf{Chapitre 3:} Après avoir présenté les différents éléments de notre expérience, nous allons nous focaliser sur les modifications apportées à notre dispositif au cours de ma thèse. En particulier, nous discuterons des deux éléments principaux de la chambre de science, la lévitation magnétique et le piège optique. Nos travaux sur ces éléments nous ont ainsi permit d'obtenir des condensats de Bose-Einstein bien meilleurs que précédemment, et de calibrer finement notre lévitation magnétique dans le but d'exploiter pleinement la dynamique de notre système. \\

\item[\textendash] \textbf{Chapitre 4:} Après avoir présenté le dispositif générant notre onde de matière, nous décrirons notre milieu désordonné. Celui-ci est issu d'un champ de tavelures laser, dont nous verrons les principales propriétés. Nous présenterons ainsi les différentes configurations de désordre que nous avons pu utiliser, et nous donnerons un aperçu de notre approche visant à dépasser les limites expérimentales auxquelles nous avons été confrontés. \\

\item[\textendash] \textbf{Chapitre 5:} Ce chapitre se focalise sur la mesure du temps de diffusion élastique d'une onde de matière dans un potentiel désordonné optique. En suivant son évolution sur un large régime de paramètres, nous pourrons identifier le régime de diffusion faible pour lequel le temps de diffusion élastique est correctement décrit par l'approximation perturbative de Born. De plus, nous étudierons de manière quantitative la pertinence du critère de désordre faible $k_{\mathrm{i}}\ls^{\mathrm{Born}}=1$, et nous verrons que celui-ci dépend fortement de la nature du désordre considéré. \\

\item[\textendash] \textbf{Chapitre 6:} Dans ce dernier chapitre, nous tacherons de décrire le comportement du temps de diffusion élastique à l'aide des fonctions spectrales. Nous verrons ainsi qu'une extension de l'approche perturbative ne permet pas de . De plus, nous verrons que l'approximation auto-consistante de Born, couramment utilisée, échoue à reproduire le comportement du temps de diffusion élastique dans un désordre de type speckle. Nous verrons donc que la connaissance fine des détails de la fonction spectrale est nécessaire pour décrire la dynamique d'une onde de matière dans un désordre speckle.
\end{itemize}


%\makeatletter
%\def\toclevel@chapter{0}
%\makeatother