\chapterimage{Fig/Anex/Logo_eksta_these.pdf}
\chapter*{Remerciements}

Difficile exercice que celui de dresser le bilan de ces quatre dernières années! Je ne peux que me sentir privilégié d'avoir eu l'occasion, l'opportunité et la chance de vivre une telle expérience, tant sur le plan scientifique et professionnel que sur le plan personnel. La richesse de cette aventure est telle qu'il m'est impossible d'établir une liste de tout ce que j'ai pu apprendre, et j'espère pouvoir en transmettre une partie par le biais de ce document. Si vous décidez de vous limiter à ces quelques lignes, c'est très bien aussi! 

Les rencontres réalisées au cours pendant ma thèse constituent selon moi une des plus grandes richesses de ces années. Ces nombreuses personnes qui m'ont fait profiter de leur talent, qui m'ont partagé leur expérience, qui m'ont témoigné leur soutien, qui m'ont accompagné, je me dois de les remercier.


Mes premières pensées sont pour mon directeur de thèse, Vincent JOSSE

Alain Aspect

Je souhaite remercier à nouveau les membres du jury d'avoir évalué les travaux que j'ai réalisés au cours de ces années. Un grand merci à Dominique DELANDE et Robin KAISER pour leur lecture attentive de ce document et les commentaires très positifs qu'ils en ont formulé, mais aussi pour les échanges que nous avons pu avoir lors de nos différentes rencontres. Merci également à Laurence PRUVOST et Jean-François CLEMENT pour leurs commentaires et leurs questions stimulantes lors de la soutenance, dont je garderai un souvenir fantastique. 

Equipe: Adrien Signoles, Vincent Denechaud, Musawwadah Mukhtar, Vasiliki Angelopoulou, Yukun Guo, Azer Trimeche, Xudong Yu

Permanents du groupe: David, Thomas, Denis, Chris, Isabelle, Marc, LSP

Thésards: Momo forever, Guillaume, Lucas, Piduss, Maxou, Ziyad, Hugo, Cécile, Antoine, Gaëtan, Julien D, Léonard, Stéphane, Olivier, Anaïs, Clémence avec qui j'ai partagé cette ultime difficulté du doctorat qu'est l'écriture de ces lignes, Romaric, Maximilian, Baptiste C, 


Service Méca: Patrick Roth et Jean-René Rullier

Service Elec: Frédéric Moron

Florence

Direction du Labo: Patrick Georges

Direction de l'EDOM: Eric Charron et Alejandro Giacomotti

Kfet: Hélène et Doussou



Cette thèse au sein de l'Institut d'Optique a également été l'occasion de participer à la vie enseignante d'une école d'ingénieurs par le biais d'une mission d'enseignement qui m'a été confiée. À ce titre, je tiens à remercier Denis BOIRON et Franck DELMOTTE pour la confiance qu'ils ont placé en moi, et j'espère avoir été à la hauteur de leurs attentes. Je souhaite également remercier le responsable de l'enseignement d'optique physique, Henri BENISTY, pour sa gentillesse, son écoute et ses conseils. Merci aussi à Julien VILLEMEJANE et Fabienne BERNARD de m'avoir encadré pour les TPs et projets d'électronique embarquée, pour nos échanges passionnants, pour tous ces conseils, ces séances d'évaluation des livrables et ces bons moments. Enfin, merci à Cédric LEJEUNE et Thierry AVIGNON pour leur aide essentielle au bon déroulement de ces TPs.

Sur un plan plus personnel, de nombreuses personnes ont été à mes côtés ces quatre dernières années


L'eksta

Les copains: Le roi Caillou et Laurine, Kazoo, Didi, 

Les colocs copernic pour les bonnes barres de rire

Les parents et la famille




