\chapterimage{Fig/Anex/Logo_eksta_these.pdf}
\chapter*{Remerciements}
\addcontentsline{toc}{chapter}{Remerciements}

Difficile exercice que celui de dresser le bilan de ces quatre dernières années! Je ne peux que me sentir privilégié d'avoir eu l'occasion, l'opportunité et la chance de vivre une telle expérience, tant sur le plan scientifique et professionnel que sur le plan personnel. La richesse de cette aventure est telle qu'il m'est impossible d'établir une liste de tout ce que j'ai pu apprendre, et j'espère pouvoir en transmettre une partie par le biais de ce document. Si vous décidez de vous limiter à ces quelques lignes, c'est très bien aussi! 

Les rencontres réalisées au cours de ma thèse constituent selon moi une des plus grandes richesses de ces années. Ces nombreuses personnes qui m'ont fait profiter de leur talent, qui m'ont partagé leur expérience, qui m'ont témoigné leur soutien, qui m'ont accompagné, je me dois de les remercier.


Mes premières pensées sont pour mon directeur de thèse, Vincent \textsc{Josse}. Je ne connaissais que peu de choses à la physique expérimentale et aux atomes froids lors de mon arrivée à l'Institut d'Optique en 2017, mais Vincent m'a fait confiance pour me familiariser et travailler avec l'expérience \textsc{Pince}. Ces quatre années sous son aile m'ont beaucoup apporté, autant en qualités scientifiques qu'en qualités humaines. Vincent a su me communiquer son goût pour la physique avec les mains, et a réussi à me sensibiliser à son extrême pédagogie par le biais de nombreuses présentations orales que j'ai eu l'occasion de faire. Somme toute, je tiens à remercier chaleureusement Vincent pour m'avoir fait tant grandir pendant ma thèse, et je ne peux qu'être fier et reconnaissant de l'avoir eu comme directeur de thèse. 





Je tiens également à remercier Alain \textsc{Aspect} pour la confiance qu'il m'a accordée, ainsi que pour les nombreuses discussions passionnantes que nous avons pu avoir. Ça a été un véritable privilège de pouvoir profiter de son expérience et de son expertise lors de nos nombreuses réunions et pour la rédaction des articles. Quatre ans après notre première rencontre, je reste admiratif de sa capacité à expliquer de manière simple les choses, et d'établir des analogies avec l'optique, dont j'ai essayé de m'inspirer pour la rédaction de ce manuscrit. 

Je souhaite remercier à nouveau les membres du jury d'avoir évalué les travaux que j'ai réalisés au cours de ces années. Un grand merci à Dominique \textsc{Delande} et Robin \textsc{Kaiser} pour leur lecture attentive de ce document et les commentaires très positifs qu'ils en ont formulé, mais aussi pour les échanges que nous avons pu avoir lors de nos différentes rencontres. Merci également à Laurence \textsc{Pruvost} et Jean-François \textsc{Clement} pour leurs commentaires et leurs questions stimulantes lors de la soutenance, dont je garderai un souvenir formidable. 

Naturellement, le contenu de ce manuscrit n'est de mon seul fait, mais est le fruit du travail acharné de toute l'équipe Transport Quantique, a.k.a. \textsc{Pince}. Je m'estime particulièrement chanceux d'avoir pu combiner différentes études menées par l'équipe, études qui ont commencé il y a maintenant plus de 10 ans! Je souhaite donc remercier les nombreux membres de l'équipe qui ont contribué à ce projet, à commencer par les très vieux thésards, Fred \textsc{Jendrzejewski}, Kilian \textsc{Müller} et Jérémie \textsc{Richard} avec lesquels j'ai eu l'occasion d'échanger. Je tiens à remercier chaleureusement Vincent \textsc{Denechaud}, pour tout le temps qu'il a passé à répondre à mes questions, aussi stupides furent-elles, pour son investissement incroyable sur la manip', mais aussi pour l'accueil qu'il m'a réservé lors de mon arrivée en 2017. J'espère avoir pu t'apprendre 2-3 choses en retour de tout ce que tu m'as apporté. Merci également à Musawwadah \textsc{Mukhtar}, dont les rires résonnent encore dans les couloirs du laboratoire, pour m'avoir appris que Digimon, ce n'est pas que des créatures, mais aussi une bande son qui envoie. Un immense merci à Adrien \textsc{Signoles}, post-doctorant arrivé peu après moi, qui m'a accompagné dans les moments difficiles de l'expérience et qui m'a beaucoup fait grandir, que ce soit personnellement ou professionnellement. Merci pour ta confiance, ta patience, et tout ce que tu as pu m'apprendre et m'apporter. Merci également à Vasiliki \textsc{Angelopoulou} qui nous a rejoint pendant un an, pour son travail d'arrache-pied et pour ses choix musicaux très qualitatifs. Merci également aux nouveaux arrivants de 2019, Azer \textsc{Trimeche} en post-doc et Yukun \textsc{Guo} en thèse, pour leur travail acharné pendant mes derniers mois de contact avec l'expérience. Merci finalement à Xudong \textsc{Yu}, arrivé en thèse en 2020. Je suis impressionné par la vitesse à laquelle cette nouvelle composition d'équipe a réussi à prendre en main l'expérience et à en tirer des résultats préliminaires très prometteurs. Je leur souhaite le meilleur pour la suite de leur thèse, qui sera éclatante, j'en suis persuadé.

Il me faut également remercier les permanents du groupe Gaz Quantiques, Denis \textsc{Boiron}, Isabelle \textsc{Bouchoule}, Thomas \textsc{Bourdel}, Marc \textsc{Cheneau}, David \textsc{Clément}, Laurent \textsc{Sanchez-Palencia} et Chris \textsc{Westbrook}, avec lesquels j'ai interagit quasi-quotidiennement. Merci à vous pour ces nombreux échanges et pour cette atmosphère d'entraide. Je suis convaincu que cette bienveillance au sein du groupe est la clé qui nous permet de trouver des solutions aux problèmes que nous rencontrons. 

Difficile de parler du groupe Gaz Quantiques sans mentionner les thésards, stagiaires et post-docs, acteurs majeurs de l'excellente ambiance du groupe. Dans le désordre (et avec quelques infiltrés), je remercie Alessandro (see you!), Alexandre, Antoine, Baptiste C, Cécile, Gaëtan (j'espère que tu te rappelles encore de mon coup de téléphone), Guillaume (Ponchoourrin!), Hugo, Julien, Léonard, Lucas, Marco, Maximilian, Maxou, Momo forever, Olivier, Piduss (tes sujets de discussion me manquent!), Stéphane et Vincent L. Mention spéciale au bureau du swagg, Anaïs, Romaric, et mes jumeaux de thèse Ziyad et Clémence (avec qui j'ai partagé cette ultime difficulté du doctorat qu'est l'écriture de ces lignes), qui m'ont accompagné, porté et supporté tous les jours. Merci à vous pour ces moments magiques qui resteront gravés dans ma mémoire. 

La réalisation de ces travaux n'aurait pas été possible sans l'aide de nombreuses personnes extérieures au groupe. Je tiens donc à remercier chaleureusement Patrick \textsc{Roth} et Jean-René \textsc{Rullier} de l'atelier mécanique pour leur travail et leur investissement sans lequel l'expérience n'aurait pas pu redémarrer. Similairement, un grand merci à Frédéric \textsc{Moron} de l'atelier d'électronique pour son aide à la réparation, l'entretien et l'amélioration de nos différents matériels. Merci également à Florence qui, même si nous n'avons pas travaillé ensemble, participe à la bonne ambiance du laboratoire. Le directeur du laboratoire, Patrick \textsc{Georges} mérite une haie d'honneur pour son suivi du laboratoire, de ses étudiants, pour sa gentillesse et son engagement auprès du personnel du laboratoire. Egalement pour leur suivi, je remercie Eric \textsc{Charron} et Alejandro \textsc{Giacomotti} à la direction de l'Ecole Doctorale qui m'ont accordé une grande partie de leur temps à la résolution de problèmes administratifs. Enfin, je souhaite remercier les personnes les plus importantes du laboratoire, Hélène et Doussou de la cafet', pour tous ces moments hilarants ou bien de totale incompréhension. 


Cette thèse au sein de l'Institut d'Optique a également été l'occasion de participer à la vie enseignante d'une école d'ingénieurs par le biais d'une mission d'enseignement qui m'a été confiée. À ce titre, je tiens à remercier Denis \textsc{Boiron} et Franck \textsc{Delmotte} pour la confiance qu'ils ont placé en moi, et j'espère avoir été à la hauteur de leurs attentes. Je souhaite également remercier le responsable de l'enseignement d'optique physique, Henri \textsc{Benisty}, pour sa gentillesse, son écoute et ses conseils. Merci aussi à Julien \textsc{Villemejane} et Fabienne \textsc{Bernard} de m'avoir encadré pour les TPs et projets d'électronique embarquée, pour nos échanges passionnants, pour tous ces conseils, ces séances d'évaluation des livrables et ces bons moments. Enfin, merci à Cédric \textsc{Lejeune} et Thierry \textsc{Avignon} pour leur aide essentielle au bon déroulement de ces TPs.

Sur un plan plus personnel, de nombreuses personnes ont été à mes côtés ces quatre dernières années. Merci à tous les copains qui m'ont amené jusque ici, Guillaume et Laurine, Dydy, Alicia, les copains de grimpe. Un énorme merci à Kévin pour ces huit dernières années de musique qualitative, de bons moments, de tapas \& sangria et j'en passe! Merci également aux colocs Copernic pour leur accueil extraordinaire et pour les moments de délire que l'on partage. 

À ma plus jolie découverte, l'Eksta, je ne sais pas comment vous remercier. Mille mercis à Alexandre, Arthur, Auriane, Bertrand, Célestine, Charlie, Chloé, Laetitia, Pierre et Rémy pour tout, pour tout ce qui est marqué dans ce carnet. 

Pour terminer, un grand merci à ma famille, que ce soient les cousins qui m'encouragent de loin ou encore mon frère Raphaël, et merci infiniment à mes parents, pour lesquels je suis très fier d'apposer le point final à cette thèse.





