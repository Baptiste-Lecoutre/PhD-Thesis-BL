\chapterimage{Fig/Anex/header_speckle.pdf}

\chapter{Propriétés spatiales du \speckle}
\label{ch:anex_speckle}




\section{Application du théorème de Wick à un \speckle }
Le théorème de Wick est un théorème particulièrement utile lors de l'étude des corrélations d'un champ de \speckle , celui-ci permettant de relier les fonctions de corrélation en intensité aux fonctions de corrélation en amplitude lorsque l'amplitude constitue une variable aléatoire complexe, gaussienne et centrée. 

L'amplitude des fluctuations d'intensité est la grandeur caractérisant l'amplitude d'un désordre \speckle, et ces dernières sont définies par $\delta I = I - \overline{I}$, où $\overline{\:\cdots\:}$ représente la valeur moyenne statistique. 

\subsection{Fonction de corrélation à deux corps des fluctuations d'intensité}
Par vœu de concision, notons $I_1=I(\mathbf{x},\lambda)$ l'intensité d'un \speckle\ à la longueur d'onde $\lambda$ et au point $\mathbf{x}=\lbrace x,y,z\rbrace$. Notons également $I_2=I(\mathbf{x}', \lambda')$.

La fonction de corrélation des fluctuations d'intensité est alors: 
\begin{align}
\overline{\delta I_1\delta I_2} &= \overline{(I_1-\overline{I_1}) (I_2-\overline{I_2})} \\
&= \overline{I_1 I_2 - I_1 \overline{I_2} - \overline{I_1} I_2 + \overline{I_1} \overline{I_2}} \\
&=\overline{I_1 I_2} - \overline{I_1}\: \overline{I_2} \text{ .}
\end{align}

Appliquons alors le théorème de Wick au terme $\overline{I_1 I_2}$ en faisant apparaître l'amplitude du champ:
\begin{align}
\overline{I_1 I_2} &= \overline{E_1 E^*_1 E_2 E^*_2} \\
&= \overline{E_1 E^*_1} \: \overline{E_2 E^*_2} + \overline{E_1 E^*_2} \: \overline{E_2 E^*_1} \\
&= \overline{I}_1 \overline{I}_2 + \left| \overline{E_1 E^*_2} \right|^2 \text{ .}
\end{align}

On détermine alors que la fonction de corrélation à deux corps des fluctuations d'intensité s'écrit sous la forme:
\begin{equation}
\overline{\delta I_1 \delta I_2} = \left| \overline{E_1 E^*_2} \right|^2 \text{ .}
\label{eq:theoreme_wick_2corps}
\end{equation}


\subsection{Fonction de corrélation à trois corps des fluctuations d'intensité}
Reprenons les notations précédentes et notons de plus $I_3=I(\mathbf{x}'',\lambda'')$. La fonction de corrélation à trois corps des fluctuations d'intensité s'écrit:
\begin{align}
\overline{\delta I_1 \delta I_2 \delta I_3} &= \overline{(I_1 -\overline{I_1})(I_2 -\overline{I_2})(I_3-\overline{I_3})} \\
&=\overline{I_1 I_2 I_3} - \overline{I_1} \: \overline{I_2} \: \overline{I_3} + \overline{I_1 \overline{I_2}\:\overline{I_3}} + \overline{\overline{I_1} I_2 \overline{I_3}} + \overline{\overline{I_1} \: \overline{I_2} I_3} \\
&\nonumber \: - \overline{I_1 I_2} \: \overline{I_3} - \overline{ I_2 I_3} \: \overline{I_1} - \overline{I_3 I_1} \: \overline{I_2} \\
&= \overline{I_1 I_2 I_3} +2 \overline{I_1}\: \overline{I_2}\: \overline{I_3} - \overline{I_1 I_2}\:\overline{I_3} - \overline{I_2 I_3}\:\overline{I_1}- \overline{I_3 I_1}\:\overline{I_2} \text{ .}
\end{align}

De même que dans la section précédente, appliquons le théorème de Wick au premier terme $\overline{I_1 I_2 I_3}$ de l'équation précédente:
\begin{align}
\overline{I_1 I_2 I_3} &= \overline{E_1 E^*_1 E_2 E^*_2 E_3 E^*_3} \\
&= \overline{E_1 E^*_1}\:\overline{E_2 E^*_2}\:\overline{E_3 E^*_3} + \overline{E_1 E^*_2}\:\overline{E_2 E^*_3}\:\overline{E_3 E^*_1} + \overline{E_1 E^*_3}\:\overline{E_2 E^*_1}\:\overline{E_3 E^*_2}\\
\nonumber & \: + \overline{E_1 E^*_1}\:\overline{E_2 E^*_3}\:\overline{E_3 E^*_2} + \overline{E_1 E^*_3}\:\overline{E_2 E^*_2}\:\overline{E_3 E^*_1} + \overline{E_1 E^*_2}\:\overline{E_2 E^*_1}\:\overline{E_3 E^*_3}\\
&= \overline{I_1}\: \overline{I_2}\: \overline{I_3} + \overline{I_1} \left|\overline{E_2 E^*_3}\right|^2 + \overline{I_2} \left|\overline{E_3 E^*_1}\right|^2 + \overline{I_3} \left|\overline{E_1 E^*_2}\right|^2 \\
&\nonumber \: +2 \mathrm{Re}\left[ \overline{E_1 E^*_2}\:\overline{E_2 E^*_3}\:\overline{E_3 E^*_1} \right] \text{ .}
\end{align}

La fonction de corrélation des fluctuations d'intensité se simplifie alors en:
\begin{align}
\overline{\delta I_1 \delta I_2 \delta I_3} &= 3 \overline{I_1}\: \overline{I_2}\: \overline{I_3} + 2\mathrm{Re}\left[ \overline{E_1 E^*_2} \: \overline{E_2 E^*_3} \: \overline{E_3 E^*_1}\right]\\
\nonumber & \: + \overline{I_1} \left( \left|\overline{E_2 E^*_3} \right|^2 -\overline{I_2 I_3}\right) + \overline{I_2} \left( \left|\overline{E_1 E^*_3}\right|^2 - \overline{I_1 I_3}\right)+ \overline{I_3} \left(\left|\overline{E_1 E^*_2}\right|^2 -\overline{I_1 I_2}\right) \\
&= 2\mathrm{Re}\left[ \overline{E_1 E^*_2} \: \overline{E_2 E^*_3} \: \overline{E_3 E^*_1}\right] \text{ .}
\label{eq:wick_3corps}
\end{align}
en appliquant le théorème de Wick à la fonction de corrélation à deux corps de l'intensité.








\section{Fonction de corrélation du diffuseur}
On s'attache ici à donner l'expression de la fonction de corrélation du diffuseur et à introduire les différents paramètres nécessaires. La géométrie du diffuseur est détaillée dans la section \ref{sc:prop_diffuseur}.

La phase localement accumulée par l'onde laser s'écrit:
\begin{equation}
\phi(\xzero)=2\pi (n-1) \frac{e(\xzero)}{\lambda} \text{ ,}
\end{equation}
où les points $\xzero=\lbrace x_0, y_0, z=0\rbrace$ sont dans le plan du diffuseur. $e(\xzero)$ correspond à l'épaisseur localement traversée, $\lambda$ est la longueur d'onde du laser et $n$ l'indice du verre. La transmission du diffuseur est aussi une grandeur aléatoire (car reliée à l'épaisseur aléatoire traversée) donnée par:
\begin{equation}
\tdiff(\xzero,\lambda)=e^{i \phi(\xzero)} \text{ .}
\end{equation}
En considérant une distribution de phase (et donc d'épaisseur) gaussienne, on peut alors obtenir une expression pour la transmission moyenne (on note $\overline{\:\cdots\:}$ la valeur moyenne sur l'ensemble des réalisations du diffuseur):
\begin{equation}
\overline{\tdiff}= \overline{e^{i \phi}} = \int{ \mathrm{d}\phi \: e^{i \phi} \: \mathcal{P}(\phi)} \quad \text{avec} \quad \mathcal{P}(\phi) = \frac{1}{\sigma_{\phi} \sqrt{2 \pi}} e^{-(\phi - \overline{\phi})^2 /2 \sigma_{\phi}^2} \text{ ,}
\end{equation}
où $\sigma_{\phi}^2$ est la variance de la distribution de phase. Il apparaît que la transmission moyenne correspond à la transformée de Fourier de la distribution de phase. On a ainsi
\begin{equation}
\overline{\tdiff} = e^{-\overline{\phi^2}/2} = e^{-\sigma_{\phi}^2/2}
\end{equation}
en choisissant $\overline{\phi}=0$. 

Dans le cas général de l'étude de la corrélation entre deux \speckles\ générés par la même réalisation du diffuseur et le même mode spatial à deux longueurs d'onde légèrement différentes, la fonction de corrélation du diffuseur est définie par la corrélation des transmissions des deux champs:
\begin{equation}
\Cdiff(\xzero,\xzero',\lambda,\lambda')=\overline{\tdiff(\xzero,\lambda)\tdiff^*(\xzero',\lambda')} \text{ .}
\end{equation}
Avec cette définition, l'étude de la fonction de corrélation d'un unique \speckle\ revient à prendre $\lambda'=\lambda$. 

Dans le cadre de l'hypothèse où les phases individuelles sont des variables aléatoires gaussiennes, la différence de phases $\phi(\xzero)-\phi'(\xzero')$ constitue également une variable aléatoire gaussienne \citep{goodman2007speckle}:
\begin{align}
\Cdiff(\xzero,\xzero',\lambda,\lambda') &= \overline{e^{i(\phi(\xzero) - \phi'(\xzero'))}} \\
&= \exp{\left[-\frac{1}{2} \overline{(\phi(\xzero)-\phi'(\xzero'))^2}\right]} \\
&= \exp{\left[ -\frac{1}{2} \overline{(\phi(\xzero)^2 + \phi'(\xzero')^2 -2 \phi(\xzero) \phi'(\xzero'))} \right] } \\
&= \exp{\left[-2\pi^2 (n-1)^2 \sigmae^2 \left( \frac{1}{\lambda^2} + \frac{1}{\lambda'^2} - \frac{2}{\lambda\lambda'} \frac{\overline{e(\xzero)e(\xzero')}}{\sigmae^2}\right)\right]} \text{ ,}
\end{align}
en faisant apparaître la fonction de corrélation de l'épaisseur et l'écart-type de la distribution d'épaisseur $\sigmae$, reliée à la largeur de la distribution de phase par $\sigma_{\phi} = 2\pi (n-1) \sigmae /\lambda$. La largeur $\re$ de la fonction de corrélation de l'épaisseur décrit la taille de la granularité transverse de la surface du diffuseur. Comme décrit section \ref{sc:prop_diffuseur}, pour une large de distribution de phases $\sigma_{\phi} \gg 2\pi$ (ou $\sigmae \gg \lambda$), on peut restreindre l'étude de la corrélation du diffuseur sur une zone petite devant $\re$. Ainsi, on peut approximer la fonction de corrélation de la granularité du diffuseur, supposée être une courbe en cloche, par:
\begin{equation}
\frac{\overline{e(\xzero)e(\xzero')}}{\sigmae^2}\approx 1- \frac{(\xzero - \xzero')^2}{2\re^2} \quad \text{pour} \quad \left| \xzero-\xzero' \right| \ll \re \text{ .}
\end{equation}

On obtient donc, pour la fonction de corrélation du diffuseur:
\begin{align}
\Cdiff(\xzero,\xzero',\lambda,\lambda') &= \exp{\left( -2 \pi^2 (n-1)^2 \sigmae^2 \left[ \frac{(\xzero - \xzero')^2}{\lambda \lambda' \re^2} + \left( \frac{1}{\lambda} - \frac{1}{\lambda'} \right)^2 \right] \right)} \\
&= \exp{\left(-\frac{{\delta\phi}^2}{2}\right)} \exp{\left(-\frac{\pi^2 \theta_{\mathrm{diff}}^2}{2} \frac{(\xzero-\xzero')^2}{\lambda \lambda'}\right)} \\
&= \exp{\left( -\frac{\delta\phi^2}{2} \right)} \exp{\left( -\frac{(\xzero-\xzero')^2}{2\rdiff\rdiff'}\right)} \text{ ,}
\end{align}
avec $\delta\phi=|\sigma_{\phi} - \sigma_{\phi'}|$ et en faisant apparaître l'angle de diffusion $\theta_{\mathrm{diff}}=\lambda /\pi \rdiff$. On retrouve ainsi la formule \ref{eq:formule_Cdiff} de la section \ref{sc:prop_diffuseur} en prenant $\lambda=\lambda'$: la fonction de corrélation du diffuseur est une gaussienne de largeur $\rdiff=\re/\sigma_{\phi}$ et de maximum 1 pour $\xzero=\xzero'$.




\section{Amplitude rayonnée}
Notons $\xd \equiv \left\lbrace x,y,d \right\rbrace$ et $\xzero \equiv \left\lbrace x_0,y_0,0 \right\rbrace$. Pour calculer le champ rayonné au point $\xd$, on utilise le principe de Huygens-Fresnel:
\begin{equation}
E(\xd)=\frac{1}{i \lambda} \int {\diff\xzero \: t(\xzero) \Ezero(\xzero) \frac{e^{ik \left| \xd-\xzero \right| }}{\left| \xd - \xzero \right| }} \text{ ,}
\label{eq:Huygens-Fresnel}
\end{equation}
avec $t(\xzero)$ la transmission du montage optique au point $\xzero$ du diffuseur, comportant l'effet du diffuseur et de la lentille. $k=2\pi /\lambda$ est le nombre d'onde.

Appliquons alors l'approximation paraxiale:
\begin{align}
\left| \xd - \xzero \right| &= \sqrt{(x-x_0)^2+(y-y_0)^2+d^2} \\
& = d \sqrt{1+\frac{(x-x_0)^2+(y-y_0)^2}{d^2}} \\
& \approx d+ \frac{(x-x_0)^2+(y-y_0)^2}{2d}
\label{eq:paraxial}
\end{align}
au premier ordre. Reportons \ref{eq:paraxial} dans \ref{eq:Huygens-Fresnel}:
\begin{equation}
E(\xd)=\frac{e^{ikd}}{i \lambda d} \int{\diff \xzero \: \tdiff(\xzero) \Ezero(\xzero) \: e^{-ik \frac{x_0^2+y_0^2}{2f}} e^{ik \frac{(x-x_0)^2+(y-y_0)^2}{2d}}} \text{ ,}
\end{equation}
où nous avons utilisé le premier ordre pour approximer l'exponentielle et l'ordre 0 pour le dénominateur. $\tdiff(\xzero)$ correspond à la transmission du diffuseur. Développons alors la dernière exponentielle complexe de cette expression:
\begin{equation}
E(\xd)=\frac{e^{i k \left( d + \frac{x^2 + y^2}{2 d}\right) }}{i \lambda d} \int{\diff \xzero \: \tdiff(\xzero) \Ezero(\xzero) \: e^{ik \frac{\xzero^2}{2 \deff}} e^{-ik\frac{\xd. \xzero}{d}}} \text{ ,}
\end{equation}
avec $1/\deff=1/d-1/f$.




%%%%%%%%%%%%%%%%%%%%%%%%%%%%%%%%%%%%%%%%%%%%%%%%%%%%%%%%%%%%%%%%%%%
\begin{comment}
\section{Fonction de corrélation}

\subsection{Calcul général}
La fonction de corrélation en amplitude s'écrit
\begin{align}
\CE(\xd,\xd',\lambda,\lambda')&=\overline{E(\xd,\lambda) E^*(\xd',\lambda')} \\
\nonumber &=\frac{e^{i \left( k (d+ \frac{x^2+y^2}{2d})-k'(d'+\frac{x'^2+y'^2}{2d'}) \right)}}{\lambda \lambda' d d'} 
&& \int{ \diff\xzero \diff\xzero' \: \overline{\tdiff(\xzero) \tdiff^*(\xzero')} \: \Ezero(\xzero) \Ezero^*(\xzero')} \\ 
& && \; e^{i\frac{k \xzero^2}{2\deff}} e^{-i\frac{k' \xzero'^2}{2\deff'}} e^{-i\frac{k(\xd.\xzero)}{d}} e^{i\frac{k'(\xd'.\xzero')}{d'}} \text{ .}
\end{align}
Appliquons alors le changement de variables $\left\lbrace \mathbf{x}_0, \mathbf{x}'_0\right\rbrace \rightarrow \left\lbrace \mathbf{x}_0, \Delta\mathbf{x}= \mathbf{x}'_0-\mathbf{x}_0\right\rbrace$ (par commodité on omettra le facteur devant l'intégrale):
\begin{align}
  \CE \propto &\int{\diff\xzero \diff\Deltax \: \Cdiff(\Deltax) \: \Ezero(\xzero)\Ezero^*(\xzero+\Deltax)} \\ 
  \nonumber & e^{i \frac{\xzero^2}{2}(\frac{k}{\deff}-\frac{k'}{\deff'})} e^{-i\frac{k'\Deltax'^2}{2\deff'}} e^{-i\frac{k'\xzero \Deltax}{\deff'}} e^{i\frac{k'\xd \Deltax}{d'}} e^{i \xzero.(\frac{k'\xd'}{d'}-\frac{k\xd}{d})} \text{ .}
\end{align}
Supposons à présent que la taille des grains du diffuseur sont très petits devant la taille typique de l'éclairement incident, c'est-à-dire qu'à l'échelle de variation de $\Cdiff$, l'éclairement incident sera constant. $\Ezero(\xzero)\Ezero^*(\xzero+\Deltax) \approx \Ezero(\xzero)\Ezero^*(\xzero)=I_0(\xzero)$. En supprimant le terme en $\Deltax^2$ supposé petit devant toutes les autres échelles de longueur, on obtient: 
\begin{equation}
\CE \propto \int{\diff\xzero  \: I_0(\xzero) \: e^{i \frac{\xzero^2}{2}(\frac{k}{\deff}-\frac{k'}{\deff'})} e^{i \xzero.(\frac{k'\xd'}{d'}-\frac{k\xd}{d})} \int{\diff\Deltax \: \Cdiff(\Deltax) \: e^{i \Deltax (\frac{k'\xd'}{d'}-\frac{k'\xzero}{\deff'})}}} \text{ .}
\end{equation}


\subsection{Expression de l'extension transverse du champ de tavelures le long de l'axe optique}
L'intensité moyenne pour un \speckle\ simple est donnée par le module carré de l'amplitude rayonnée, qui peut se réécrire à l'aide de la fonction de corrélation en amplitude:
\begin{align}
\overline{I(\xd)} &= \overline{E(\xd,\lambda)E^*(\xd,\lambda)} \\
&\propto \int{\diff\xzero \: I_0(\xzero) \: \int{\diff\Deltax \: \Cdiff(\Deltax) \: e^{i\Deltax. \left( \frac{k\xd}{d}-\frac{k\xzero}{\deff}\right)}}} \\
&\propto \int{\diff\Deltax \: \Cdiff(\Deltax) \: e^{i \Deltax. \frac{k\xd}{d}} \: \int{\diff\xzero \: I_0(\xzero) \: e^{-i \xzero .\frac{k\Deltax}{\deff}}}} \\
&\propto \widetilde{\Cdiff}(\frac{k\xd}{d}) \ast I_0(\frac{\xd\deff}{d}) \text{ ,}
\end{align}
où $\ast$ dénote le produit de convolution et $\widetilde{\Cdiff}(k\xd/d)=\int{\diff\Deltax\:\Cdiff(\Deltax)\exp{(ik\Deltax.\xd/d)}}$ est la transformée de Fourier de la fonction de corrélation de la transmission du diffuseur. On retrouve ainsi le résultat \ref{eq:evolution_extension_transverse_speckle} du chapitre \ref{ch:Speckle}.

Aux alentours du plan de Fourier, $\deff\rightarrow\infty$ et donc on peut assimiler la contribution de l'intensité incidente $I_0(\xf\deff/d)$ à une distribution de Dirac $\delta(\xf)$ de telle sorte que l'intensité moyenne soit donnée par
\begin{equation}
\overline{I(\xf)} \propto \widetilde{\Cdiff} (\frac{k\xf}{d}) \text{ .}
\end{equation}





\section{Corrélations transverse et longitudinales d'un \speckle\ monochromatique}% et bichromatiques}
\subsection{Corrélation transverse le long de l'axe optique}
Considérons le cas d'un unique \speckle , que l'on étudie dans un plan orthogonal à l'axe optique. Posons $\lambda = \lambda'$, $d=d'$ et $\xd'=\lbrace 0,0,d \rbrace$:
\begin{align}
\CE(x,y,d)&=\overline{E(x,y,d)E^*(0,0,d)}\\
&\propto \int{\diff\xzero \: I_0(\xzero) \: e^{-ik\frac{\xd .\xzero}{d}} \int{\diff\Deltax \: \Cdiff(\Deltax)\: e^{-ik\frac{\Deltax.\xzero}{\deff}}}}\\
&\propto \widetilde{I_0}\left(\frac{k\xd}{d}\right) \ast \Cdiff \left( \frac{\deff}{d} \xd\right) \text{ .}
\end{align}
On retrouve bien le résultat \ref{eq:correlation_transverse_speckle} à l'aide de l'invariance par translation selon les directions transverses. 

Aux alentours du plan de Fourier, $\deff\rightarrow\infty$ et donc on peut assimiler $\Cdiff(\xf\deff/d)$ à une distribution de Dirac $\delta(\xf)$. On obtient alors que la fonction de corrélation en amplitude est donnée par
\begin{equation}
\CE(\xf)\propto \widetilde{I_0}\left(\frac{k\xf}{d}\right) \text{ .}
\end{equation}


\subsection{Corrélation longitudinale autour du plan de Fourier}
Plaçons nous sur l'axe optique, autour du plan de Fourier, et posons $\lambda=\lambda'$. Pour $\xd=\lbrace 0,0,d \rbrace=0$, et $\xd'=\xf=\lbrace 0,0,f \rbrace$, la fonction de corrélation longitudinale:
\begin{align}
\CE&=\overline{E(0,0,d)E^*(0,0,f)} \\
& \propto \int{\diff\xzero \: I_0(\xzero) \: e^{i\frac{\xzero^2 k}{2 \deff}} \int{\diff\Deltax \: \Cdiff(\Deltax)}}
\end{align}
Proche du plan de Fourier, on pose $d=f+\delta z$, donc $1/\deff=1/d-1/f\approx-\delta z/f^2$. Finalement, on obtient:
\begin{equation}
\CE \propto \int{\diff\xzero \: I_0(\xzero) \: e^{-i\frac{\delta z k \xzero^2}{2f^2}}} \text{ .}
\end{equation}
On retrouve l'expression \ref{eq:correlation_longitudinale_1_speckle} du chapitre \ref{ch:Speckle}.










\subsection{Corrélation tridimensionnelle aux alentours du plan de Fourier}
Il est possible de combiner les résultats précédents sous la forme d'une fonction de corrélation tridimensionnelle aux alentours du plan de Fourier. Considérons le cas $\lambda=\lambda'$, $\xd=\lbrace x,y,d=f+\delta z\rbrace$ et $\xd'=\xf=\lbrace 0,0,f \rbrace$. Alors, 
\begin{align}
\CE(x,y,\delta z)&\propto \int{\diff\xzero \: I_0(\xzero) \: e^{-ik \frac{\xzero^2 \delta z}{2 f^2}} \: e^{-ik \frac{\xzero.\xd}{f}}} \int{\diff\Deltax \: \Cdiff(\Deltax)}\\
&\propto \mathrm{TF}{\left[ I_0(\xzero) e^{-ik\xzero^2 \delta z/2f^2} \right]}_{\frac{k\xd}{f}} \text{ .}
\label{eq:correlation_3D_monochromatique_annexe}
\end{align}
On retrouve donc la fonction de corrélation $c_{\mathrm{3D}}$ de la formule \ref{eq:correlation_3D_paraxial_effectif} en prenant le module carré de l'expression \ref{eq:correlation_3D_monochromatique_annexe}.











\section{Corrélation d'un \speckle\ bichromatique}
Regardons maintenant les corrélations entre les champs de même intensité aux deux longueurs d'onde, mais à la même position physique. Posons $\lambda\neq\lambda'$ et $\xd=\xd'=\lbrace x,y,d=f+\delta z \rbrace$:
\begin{equation}
\CE(\xd,\lambda,\lambda') \propto \int{\diff\xzero \: I_0(\xzero) \: e^{i\frac{\xzero^2}{2\deff}(k-k')} \: e^{i\frac{\xzero.\xd}{d}(k'-k)} \int{\diff\Deltax \: \Cdiff(\Deltax) \: e^{i k' \Deltax (\frac{\xd}{d}-\frac{\xzero}{\deff})}}} \text{ ,}
\end{equation}
et dans la limite où $D\gg \rdiff$, on peut assimiler la fonction de corrélation $\Cdiff(\Deltax)$  à une distribution infiniment fine $\Cdiff(0) \delta(\Deltax)$. La fonction de corrélation devient alors
\begin{equation}
\CE(\xd,\lambda,\lambda') \propto  \Cdiff(0) \int{\diff\xzero \: I_0(\xzero) \: e^{i\frac{\xzero^2}{2\deff}(k-k')} \: e^{i\frac{\xzero.\xd}{d}(k'-k)}} \text{ .}
\end{equation}
En posant $\lambda'=\lambda+ \delta \lambda$, on obtient
\begin{align}
\CE(\xd,\lambda, \lambda+\delta\lambda) &\propto \Cdiff(0) \int{\diff\xzero \: I_0(\xzero) \: e^{-ik \frac{\xzero^2 \delta z}{2f^2} \frac{\delta \lambda}{\lambda}} \: e^{-ik \frac{\xzero.\xd}{f}\frac{\delta \lambda}{\lambda}}} \\
& \propto e^{-\delta\phi^2/2} \times \mathrm{TF}{\left[ I_0(\xzero)\: e^{-ik\frac{\xzero^2 \delta z}{2f^2} \frac{\delta\lambda}{\lambda}} \right]}_{\frac{k \xd}{f}\frac{\delta \lambda}{\lambda}} \text{ ,}
\end{align}
où l'on reconnaît la fonction de corrélation \ref{eq:correlation_3D_monochromatique_annexe} avec un facteur d'échelle $\lambda/\delta\lambda$ dans les trois directions de l'espace.


%%%% end
\end{comment}
%%%%%%%%%%%%%%%%%%%%%%%%%%%%%%%%%%%%%%%%%%%%%%%%%%%%%%%%%%%%%%%%%%%%%%%%%%%














\section{Calcul complet de la fonction de corrélation en amplitude à deux corps}
Notons $\xd=\lbrace x,y,z=d\rbrace$ et $\xd'=\lbrace x',y',z=d'\rbrace$. L'outil de base permettant de décrire les propriétés spatiales d'un champ de \speckle\ est la fonction de corrélation à deux corps de l'amplitude rayonnée. Celle-ci s'écrit:
\begin{align}
\overline{E(\xd,\lambda)E^*(\xd',\lambda')} &= \frac{e^{ik \left( d+\frac{x^2+y^2}{2d} \right)}}{\lambda d} \frac{e^{-i k' \left( d'+ \frac{x'^2+y'^2}{2d'} \right)}}{\lambda' d'} \\
\nonumber & \quad \times \int{\diff\xzero \diff\xzero' \: \overline{\tdiff(\xzero, \lambda)\tdiff^*(\xzero', \lambda')} \Ezero(\xzero,\lambda) \Ezero^*(\xzero',\lambda')} \\
&\nonumber \quad \times e^{ik\frac{ \xzero^2}{2\deff}} e^{-ik'\frac{ \xzero'^2}{2\deff'}} e^{-ik \frac{\xd . \xzero}{d}} e^{ik' \frac{\xd' . \xzero'}{d'}} \text{ ,}
\end{align}
dans le cas général d'un éclairement polychromatique, et où le moyennage d'ensemble $\overline{\:\cdots\:}$ n'agit que sur les transmissions $\tdiff$.

Effectuons alors le changement de variables $\lbrace \xzero,\xzero' \rbrace\rightarrow \lbrace \xc=(\xzero+\xzero')/2 ,\: \Deltax=\xzero'-\xzero \rbrace$. En réarrangeant les termes de phase, on peut réécrire l'expression précédente sous la forme:
\begin{align}
\overline{E(\xd,\lambda)E^*(\xd',\lambda')} &= \frac{e^{ik\left( d + \frac{x^2+y^2}{2d} \right)} e^{-ik' \left( d' + \frac{x'^2+y'^2}{2d'}\right)}}{\lambda \lambda' d d'} \\
\nonumber & \times \int{\diff\xc \diff\Deltax \: \Cdiff(\Deltax,\lambda,\lambda') \Ezero(\xc-\Deltax/2,\lambda)\Ezero^*(\xc+\Deltax/2,\lambda')} \\
& \nonumber \times e^{i\frac{\xc^2}{2} (\frac{k}{\deff} - \frac{k'}{\deff'})} e^{-i \frac{\Deltax . \xc}{2} ( \frac{k}{\deff} + \frac{k'}{\deff'})} e^{i \frac{\Deltax^2}{8}(\frac{k}{\deff} - \frac{k'}{\deff'})} \\
&\nonumber \times  e^{i\xc.(\frac{k' \xd'}{d'} - \frac{k \xd}{d})} e^{i\frac{\Deltax}{2} . (\frac{k \xd}{d} + \frac{k' \xd'}{d'})} \text{.}
\end{align}

Supposons à présent que la taille des grains du diffuseur est très petite devant la taille typique de l'éclairement incident, c'est-à-dire qu'à l'échelle de variation de $\Cdiff$, l'éclairement incident est constant: $\Ezero(\xc-\Deltax/2,\lambda) \Ezero^*(\xc+\Deltax/2,\lambda') \approx \Ezero(\xc,\lambda) \Ezero^*(\xc,\lambda') = I_0(\xc)$ en supposant également que les champs sont de même intensité. En supprimant le terme de phase en $\Deltax^2$ supposé petit devant toutes les autres échelles de longueur, on obtient finalement l'expression:
\begin{align}
\overline{E(\xd,\lambda)E^*(\xd',\lambda')} &= \frac{e^{ik\left( d + \frac{x^2+y^2}{2d} \right)} e^{-ik' \left( d' + \frac{x'^2+y'^2}{2d'}\right)}}{\lambda \lambda' d d'} \\
\nonumber & \quad \times \int{\diff\xc \: I_0(\xc) e^{i\frac{\xc^2}{2}(\frac{k}{\deff} - \frac{k'}{\deff'})} e^{i \xc. (\frac{k' \xd'}{d'}-\frac{k \xd}{d})}} \\
\nonumber & \quad \times \int{\diff\Deltax \: \Cdiff(\Deltax,\lambda,\lambda') \: e^{-i\frac{\Deltax . \xc}{2}(\frac{k}{\deff} + \frac{k'}{\deff'})} e^{i\frac{\Deltax}{2}. (\frac{k \xd}{d} +\frac{k' \xd'}{d'})}} \text{ .}
\label{eq:fonction_correlation_generale}
\end{align}




\section{Expression du profil d'intensité moyenne du champ de tavelures le long de l'axe optique}
L'intensité moyenne d'un champ de tavelures monochromatique est donnée par le module carré de l'amplitude rayonnée, qu'il est possible d'exprimer à l'aide de la fonction de corrélation en amplitude calculée précédemment:
\begin{align}
\overline{I(\xd)} &= \overline{E(\xd,\lambda) E^*(\xd,\lambda)} \\
&= \frac{1}{\lambda^2 d^2} \int{\diff\xc \: I_0(\xc) \int{\diff\Deltax \: \Cdiff(\Deltax) \: e^{-i \Deltax . \xc \frac{k}{\deff}} \: e^{i\Deltax \frac{k \xd}{d}}}} \\
&= \frac{1}{\lambda^2 d^2} \int{\diff\xc \: I_0(\xc) \: \widetilde{\Cdiff}\left(\frac{k\xd}{d} - \frac{k\xc}{\deff}\right)} \text{ ,}
\end{align}
en faisant apparaître $\widetilde{\Cdiff}(\boldsymbol{\xi})= \int{\diff\Deltax \: \Cdiff(\Deltax) \: e^{i\Deltax\boldsymbol{\xi}}}$ la transformée de Fourier de la fonction de corrélation de la transmission du diffuseur.

Procédons maintenant au changement de variable $\mathbf{p}_{\mathrm{c}} = \xc d/\deff$:
\begin{align}
\overline{I(\xd)} &= \frac{1}{\lambda^2d^2} \int{\diff\mathbf{p}_{\mathrm{c}} \frac{\deff^2}{d^2} I_0\left( \mathbf{p}_{\mathrm{c}} \frac{\deff}{d}\right) \: \widetilde{\Cdiff} \left( \frac{k}{d} [\xd - \mathbf{p}_{\mathrm{c}}] \right)} \\
&= \frac{1}{\lambda^2 d^2} \frac{\deff^2}{d^2} I_0\left( \xd \frac{\deff}{d}\right) \ast \widetilde{\Cdiff}\left( \frac{k\xd}{d}\right) \text{ ,}
\end{align}
où le symbole $\ast$ dénote le produit de convolution. On retrouve ainsi le résultat \ref{eq:evolution_extension_transverse_speckle} annoncé au chapitre \ref{ch:Speckle}.

Aux alentours du plan de Fourier, $\deff\rightarrow\infty$ et il est alors possible d'assimiler la contribution de l'intensité incidente $I_0(\xf \deff/d) \times \deff^2/d^2$ à une distribution de Dirac $\delta(\xf)$ de telle sorte que l'intensité moyenne soit donnée par
\begin{equation}
\overline{I(\xf)} \propto \widetilde{\Cdiff}\left( \frac{k\xf}{f}\right) \text{ .}
\end{equation}




\section{Fonctions de corrélation d'un \speckle\ monochromatique}
\subsection{Corrélation transverse le long de l'axe optique}
Intéressons-nous à présent à la forme de la fonction de corrélation transverse d'un \speckle\ monochromatique en fonction de la distance $d$ au diffuseur. Cette fonction décrit la corrélation entre deux points situés dans le même plan orthogonal à l'axe optique et proches de ce dernier. Notons $\xd=\lbrace x,y,z=d\rbrace$ et $\zerod=\lbrace 0,0,d \rbrace$, la fonction de corrélation en amplitude recherchée s'écrit alors:
\begin{align}
\overline{E(\xd,\lambda)E^*(\zerod,\lambda)} &= \frac{e^{i k (x^2+y^2)/2d}}{\lambda^2 d^2} \int{\diff\xc \: I_0(\xc) \: e^{-i k \frac{\xc .\xd}{d}}} \\
&\nonumber \quad \times \int{\diff\Deltax \: \Cdiff(\Deltax) \: e^{-ik \frac{\Deltax . \xc}{\deff}} \: e^{ik \frac{\Deltax . \xd}{2d}}} \text{ ,}
\end{align}
d'après l'équation \ref{eq:fonction_correlation_generale}.

En supposant que les grains du diffuseur sont petits devant la taille typique de l'éclairement incident, il est possible de négliger la dernière exponentielle complexe et on obtient:
\begin{equation}
\overline{E(\xd,\lambda)E^*(\zerod,\lambda)} = \frac{e^{i k (x^2+y^2)/2d}}{\lambda^2 d^2} \int{\diff\xc \: I_0(\xc) \: e^{-i k \frac{\xc .\xd}{d}}}  \int{\diff\Deltax \: \Cdiff(\Deltax) \: e^{-ik \frac{\Deltax . \xc}{\deff}} } \text{ ,}
\end{equation}
puis intervertissons les intégrales:
\begin{align}
\overline{E(\xd,\lambda)E^*(\zerod,\lambda)}&=\frac{e^{ik (x^2+y^2)/2d}}{\lambda^2 d^2} \int{\diff\Deltax \: \Cdiff(\Deltax) \int{\diff\xc \: I_0(\xc) \: e^{-i\frac{k\xc}{d} \left(\xd+\frac{\Deltax d}{\deff}\right)}}} \\
&= \frac{e^{ik(x^2+y^2)/2d}}{\lambda^2 d^2} \int{\diff\Deltax \: \Cdiff(\Deltax) \: \widetilde{I_0}\left( \frac{k}{d}\left[ \xd + \frac{\Deltax d}{\deff} \right]\right)} \text{ ,}
\end{align}
en faisant apparaître $\widetilde{I_0}(\boldsymbol{\xi})=\int{\diff\xc \: I_0(\xc) e^{i\xc.\boldsymbol{\xi}}}$ la transformée de Fourier de l'éclairement incident grâce au changement de variable $\xc\rightarrow - \xc$.

Procédons maintenant à un nouveau changement de variable $\Delta\mathbf{p}=\Deltax d/\deff$. La fonction de corrélation se réécrit:
\begin{align}
\overline{E(\xd,\lambda) E^*(\zerod,\lambda)}&= \frac{e^{ik (x^2+y^2)/2d}}{\lambda^2 d^2} \int{\diff\Delta\mathbf{p} \:\frac{\deff^2}{d^2} \widetilde{I_0}\left( \frac{k}{d}[\xd+\Delta\mathbf{p}]\right) \: \Cdiff\left( \Delta\mathbf{p}\frac{\deff}{d}\right)}\\
&=\frac{e^{ik (x^2+y^2)/2d}}{\lambda^2 d^2} \frac{\deff^2}{d^2} \widetilde{I_0}\left(\xd\frac{\deff}{d} \right) \ast \Cdiff\left(\frac{k\xd}{d}\right) \text{ ,}
\end{align}
où le symbole $\ast$ dénote le produit de convolution. On retrouve ainsi le résultat \ref{eq:correlation_transverse_speckle} annoncé au chapitre \ref{ch:Speckle}. 

Aux alentours du plan de Fourier, $\deff\rightarrow\infty$ et on peut donc assimiler la contribution de la corrélation du diffuseur $\Cdiff(\xf\deff/d) \times \deff^2/d^2$ à une distribution de Dirac $\delta(\xf)$. On détermine alors que la fonction de corrélation transverse en amplitude aux alentours du plan de Fourier est déterminée par la transformée de Fourier de l'éclairement incident:
\begin{equation}
\overline{E(\xf,\lambda)E^*(\zerof,\lambda)} \propto \widetilde{I_0}\left(\frac{k\xf}{f} \right) \text{ .}
\end{equation}



\subsection{Corrélation tridimensionnelle aux alentours du plan de Fourier}
Notons $\mathbf{x}_{\perp}=\lbrace x,y\rbrace$, $\Deltax_{\perp}=\lbrace \Delta x, \Delta y\rbrace$ et $\mathbf{0}=\lbrace 0,0 \rbrace$ dans le plan transverse $\lbrace x,y \rbrace$. La fonction de corrélation à deux corps normalisée décrivant les propriétés spatiales d'un \speckle\ monochromatique de longueur d'onde $\lambda$ à trois dimensions est définie par 
\begin{equation}
c_{\mathrm{3D}}(\Deltax_{\perp},\delta z)=\frac{\overline{\delta I(\mathbf{x}_{\perp} + \Deltax_{\perp}, d+\delta z) \delta I(\mathbf{x}_{\perp}, d)}}{\overline{\delta I^2}} \text{.,}
\end{equation}
Aux alentours du plan de Fourier ($d\approx f$ et $\delta z \ll \delta f$) et proche de l'axe optique ($|\mathbf{x}_{\perp}|\ll \speckleext$), le numérateur de cette fonction de corrélation s'écrit $\overline{\delta I(\Delta\mathbf{x}_{\perp},f+\delta z) \delta I(\mathbf{0}, f)}$ et peut être calculé à l'aide du théorème de Wick \ref{eq:theoreme_wick_2corps}. Cette nouvelle fonction de corrélation vaut:
\begin{align}
\overline{\delta I (\Deltax_{\perp}, f+\delta z) \delta I(\mathbf{0},f)} &= \left| \overline{E(\Deltax_{\perp},f+\delta z)E^*(\mathbf{0},f)} \right|^2 \\
&= \left| \frac{1}{\lambda^2 f^2} \int{\diff\xc \: I_0(\xc) \: e^{-i\frac{\xc^2 k \delta z}{2f^2}} \: e^{-i\frac{k \xc. \Deltax_{\perp}}{f}}}\right. \\
&\nonumber \quad\times \left.\int{\diff\Deltax \: \Cdiff(\Deltax) \: e^{i\frac{\Deltax . \xc k \delta z}{2 f^2}} \: e^{i\frac{k\Deltax . \Deltax_{\perp}}{2f}}} \right|^2 \text{ .}
\end{align}

Comparons alors les différents termes de phase. Toujours en supposant que la taille des grains du diffuseur est très petite devant la taille typique de l'éclairement incident, c'est-à-dire $\Deltax \ll \xc$, il est possible de négliger la contribution des exponentielles complexes de la seconde intégrale. L'expression de la fonction de corrélation se réduit alors à
\begin{align}
{\left| \overline{E(\Deltax_{\perp},f+\delta z)E^*(\mathbf{0},f)}\right|}^2 &= {\left| \frac{1}{\lambda^2 f^2}  \int{\diff\Deltax \: \Cdiff(\Deltax)} \int{\diff\xc \: I_0(\xc) \: e^{-i\frac{\xc^2 k \delta z}{2f^2}} \: e^{-i \frac{k \xc . \Deltax_{\perp}}{f}}} \right|}^2 \\
&= \frac{1}{\lambda^4 d^4} \left|\int{\diff\Deltax \: \Cdiff(\Deltax)}\right|^2 \: \left|\int{\diff\xc \: I_0(\xc) \: e^{-i\frac{\xc^2 k\delta z}{2f^2}} \: e^{-i\frac{k\xc.\Deltax_{\perp}}{f}}} \right|^2 \text{ .}
\end{align}

Il est à présent possible de déterminer la fonction de corrélation tridimensionnelle $c_{\mathrm{3D}}(\Deltax_{\perp}, \delta z)$ en normalisant l'expression précédente par $|\overline{E(\mathbf{0},f)E^*(\mathbf{0},f)}|^2$. On obtient finalement
\begin{equation}
c_{\mathrm{3D}}(\Delta \mathbf{x}_{\perp} , \delta z) = \frac{{\left| \mathrm{TF} \left[ I_0(\xc)\: e^{-i\frac{\xc^2 k \delta z}{2f^2}} \right]_{\frac{k\Deltax_{\perp}}{f}}\right|}^2}{{\left|\displaystyle\int{\diff\xc \: I_0(\xc)}\right|}^2} \text{ ,}
\end{equation}
%\begin{equation}
%c_{\mathrm{3D}}(\Delta \mathbf{x}_{\perp} , \Delta z) = \frac{{\left| \displaystyle\int{\diff\xc \: I_0(\xc) \: e^{-i\frac{\xc^2 k \Delta z}{2 f^2}} \: e^{-i\frac{k \xc . \Delta \mathbf{x}_{\perp}}{f}}}\right|}^2}{{\left|\displaystyle\int{\diff\xc \: I_0(\xc)}\right|}^2} \text{ ,}
%\end{equation}
correspondant à l'équation \ref{eq:correlation_3D_paraxial_effectif} du chapitre \ref{ch:Speckle}.

\paragraph*{Corrélation longitudinale aux alentours du plan de Fourier}
La fonction de corrélation longitudinale aux alentours du plan de Fourier présentée dans la section \ref{sc:correlation_longitudinale} apparaît donc comme un cas particulier de la fonction de corrélation tridimensionnelle calculée dans la section précédente. 

En se plaçant sur l'axe optique et aux alentours du plan de Fourier dans le cas d'un \speckle\ monochromatique, on tire directement de l'équation \ref{eq:fonction_correlation_generale} que:
\begin{equation}
\overline{E(f+\delta z)E^*(f)} \propto \int{\diff\xc \: I_0(\xc) \: e^{-ik \frac{\xc^2 \delta z}{2f^2}}} \text{ ,}
\end{equation}
ce qui constitue le résultat \ref{eq:correlation_longitudinale_1_speckle} du chapitre \ref{ch:Speckle}.

\subsection{Détermination de la fonction de corrélation à trois corps}
La fonction de corrélation à trois corps des fluctuations d'intensité peut être exprimée à l'aide d'une permutation de fonctions de corrélation à deux corps de l'amplitude rayonnée à l'aide du théorème de Wick selon l'équation \ref{eq:wick_3corps}. Dans cette section, nous nous attachons à déterminer cette fonction de corrélation à trois corps dans le plan de Fourier.

Comme explicité dans la section \ref{sc:speckle_non_paraxial}, la fonction de corrélation transverse à deux corps des fluctuations d'intensité $c_{\mathrm{2D}}$ est très bien décrite par une gaussienne dont la demi-largeur à $1/e$ est donnée par $\sigmap=\SI{0.5}{\micro\metre}$. Il est donc possible d'écrire que
\begin{equation}
\left|\overline{E(\xf,\lambda)E^*(\zerof,\lambda)}\right|=\left|\overline{E(\zerof,\lambda)E^*(\zerof,\lambda)} \right| \times e^{-\xf^2/2\sigmap^2} \text{ .}
\end{equation}

De plus, les calculs de fonction de corrélation à deux corps en amplitude menés dans la section précédente montrent que la phase de la fonction de corrélation à trois corps s'annule par circularité. On trouve alors que la fonction de corrélation à trois corps normalisée s'exprime:
\begin{align}
c_{\mathrm{3I}}(\mathbf{x}_1,\mathbf{x}_2,\mathbf{x}_3) &= \frac{\overline{\delta I(\mathbf{x}_1) \delta I(\mathbf{x}_2) \delta I(\mathbf{x}_3)}}{\overline{I}^3}\\
&= 2 e^{-\frac{(\mathbf{x}_2-\mathbf{x}_1)^2}{2\sigmap^2}} \: e^{-\frac{(\mathbf{x}_3-\mathbf{x}_2)^2}{2\sigmap^2}} \: e^{-\frac{(\mathbf{x}_1-\mathbf{x}_3)^2}{2\sigmap^2}} \text{ .}
\end{align}






\section{Fonction de corrélation d'un \speckle\ bichromatique}
\subsection{Cas d'un spectre bichromatique étroit}
Intéressons-nous à présent à la corrélation entre deux champs de \speckle\ générés par le même diffuseur, à deux longueurs d'onde $\lambda$ et $\lambda'$ légèrement différentes, et observés à la même position physique $\lbrace \mathbf{x}_{\perp}, d=f+\delta z \rbrace$.


La fonction de corrélation normalisée décrivant les similitudes entre deux champs de tavelures réalisés par le même diffuseur à deux longueurs d'onde est définie par
\begin{equation}
c_{2\lambda}(\mathbf{x}_{\perp},d,\lambda, \lambda')=\frac{\overline{\delta I (\mathbf{x}_{\perp},d,\lambda) \delta I (\mathbf{x}_{\perp},d,\lambda')}}{\overline{I(\mathbf{x}_{\perp},d,\lambda)}\:\overline{I(\mathbf{x}_{\perp},d,\lambda')}}
\end{equation}

La fonction de corrélation des fluctuations d'intensité apparaissant au numérateur peut être obtenue à l'aide du théorème de Wick et vaut:
\begin{align}
\left| \overline{E(\mathbf{x}_{\perp},f+\delta z,\lambda)E^*(\mathbf{x}_{\perp},f+\delta z,\lambda')} \right|^2 &= \left| \frac{1}{\lambda\lambda' d^2} \int{\diff\xc \: I_0(\xc) \: e^{-i\frac{\xc^2 \delta z}{2f^2}(k-k')} \: e^{i \frac{\mathbf{x}_{\perp}.\xc}{f}(k'-k)}}\right. \\
&\nonumber \quad\times \left. \int{\diff\Deltax \: \Cdiff(\Deltax, \lambda, \lambda')\: e^{i \frac{\Deltax.\xc \delta z}{f^2}\frac{k+k'}{2}} \: e^{i\frac{\Deltax.\mathbf{x}_{\perp}}{f}\frac{k+k'}{2}}}\right|^2 \text{ .}
\end{align}

Posons $\lambda'=\lambda+\delta\lambda$, avec $\delta\lambda\ll\lambda$. Nous avons alors
\begin{align}
\frac{k+k'}{2} &\approx \frac{2\pi}{\lambda} =k \quad \text{et}\\
k-k' &\approx \frac{2\pi}{\lambda} \frac{\delta\lambda}{\lambda} = k \frac{\delta\lambda}{\lambda}
\end{align}
au premier ordre.

Restreignons à présent l'étude à un domaine proche du centre de la figure de \speckle , de telle sorte que $\mathbf{x}_{\perp}\ll\sigma_{\mathrm{ex}}$ et $\delta z=d-f \ll \delta f$. Ces conditions se traduisent par 
\begin{align}
|\mathbf{x}_{\perp}| &\ll \frac{f}{k |\Deltax|}  \quad\text{et}\\
\delta z &\ll \frac{f^2}{k|\xc||\Deltax|} \text{ ,}
\end{align}
qui permettent alors de négliger les exponentielles complexes de la seconde intégrale.

La fonction de corrélation bichromatique de simplifie alors en:
\begin{align}
\left| \overline{E(\mathbf{x}_{\perp},f+\delta z,\lambda)E^*(\mathbf{x}_{\perp},f+\delta z,\lambda')} \right|^2 &= \left| \frac{1}{\lambda \lambda' f^2} \int{\diff\xc \: I_0(\xc) \: e^{-ik\frac{\xc^2 \delta z}{2f^2} \frac{\delta\lambda}{\lambda}} \: e^{ik \frac{\mathbf{x}_{\perp}.\xc}{f}\frac{\delta\lambda}{\lambda}}} \right. \\
\nonumber & \qquad \left. \times \int{\diff\Deltax \: \Cdiff(\Deltax,\lambda,\lambda')} \right|^2 \\
&= \frac{1}{\lambda^2 \lambda'^2 f^4} \left| \int{\diff\Deltax \: \Cdiff(\Deltax,\lambda,\lambda')} \right|^2 \\
\nonumber & \qquad \times\left|\int{\diff\xc \: I_0(\xc) \: e^{-ik\frac{\xc^2 \delta z}{2f^2} \frac{\delta\lambda}{\lambda}} \: e^{ik\frac{\mathbf{x}_{\perp}.\xc}{f}\frac{\delta\lambda}{\lambda}}} \right|^2
\end{align}

Rappelons à présent que l'intensité moyenne proche peut être déterminée et son expression aux alentours du centre du champ de \speckle\ est donnée par:
\begin{equation}
\overline{I}(\lambda)= \frac{1}{\lambda^2 f^2} \int{\diff\xc \: I_0(\xc)} \int{\diff\Deltax \: \Cdiff(\Deltax,\lambda)} \text{ ,}
\end{equation}
que nous pouvons utiliser afin de calculer la fonction de corrélation bichromatique normalisée. Cette dernière devient donc:
\begin{align}
c_{2\lambda}(\mathbf{x}_{\perp},\delta z,\lambda,\lambda') &= \frac{\left| \displaystyle\int{\diff\Deltax \: \Cdiff(\Deltax,\lambda,\lambda')}\right|^2}{\displaystyle\int{\diff\Deltax \: \Cdiff(\Deltax,\lambda)}\displaystyle\int{\diff\Deltax \: \Cdiff(\Deltax,\lambda')}} \\
&\nonumber \quad\times \frac{\left|\displaystyle\int{\diff\xc \: I_0(\xc) \: e^{-ik \frac{\xc^2 \delta z}{2f^2} \frac{\delta\lambda}{\lambda}} \: e^{ik\frac{\mathbf{x}_{\perp}.\xc}{f} \frac{\delta\lambda}{\lambda}}} \right|^2}{\left|\displaystyle\int{\diff\xc \: I_0(\xc)} \right|^2} \text{ ,}
\end{align}
et fait apparaître deux termes, le premier caractérisant une décorrélation liée à la propagation dans le diffuseur, et le second lié à un déphasage différent lors de la propagation en espace libre après diffraction.

Le premier terme peut être calculé en utilisant le fait que:
\begin{align}
\int{\diff\Deltax \: \Cdiff(\Deltax ,\lambda)} &= \int{\diff\Deltax \: e^{-\Deltax^2/2\rdiff^2}} \\
&= 2\pi \rdiff^2 \text{ ,}
\end{align}
dans le cas d'un \speckle\ monochromatique.

On peut finalement déterminer complètement la fonction de corrélation bichromatique, dont l'écriture devient:
\begin{align}
c_{2\lambda}(\mathbf{x}_{\perp},\delta z,\lambda,\delta\lambda) &= \Cdiff(\mathbf{0},\lambda,\delta\lambda) \frac{\left| \mathrm{TF} \left[ I_0(\xc) \: e^{-ik\frac{\xc^2 \delta z}{2f^2}}\right]_{\frac{k\mathbf{x}_{\perp}}{f} \frac{\delta\lambda}{\lambda}}\right|^2}{\left|\displaystyle\int{\diff\xc \: I_0(\xc)} \right|^2}\\
&= e^{-\delta\phi^2} c_{\mathrm{3D}}\left(\mathbf{x}_{\perp} \frac{\delta\lambda}{\lambda}, \delta z \frac{\delta\lambda}{\lambda}\right) \text{ ,}
\end{align}
en reconnaissant l'expression de la fonction de corrélation tridimensionnelle d'un \speckle\ monochromatique. On retrouve ainsi le résultat \ref{eq:correlation_2_lambda} annoncé dans le chapitre \ref{ch:Speckle}.



\subsection{Extension au cas d'un spectre arbitraire}
Il est possible d'étendre l'étude menée précédemment au cas d'un spectre de forme arbitraire. L'intensité totale $I$ résulte alors de l'ensemble des rayonnements émis à toutes les fréquences:
\begin{equation}
I=\int{\diff\nu \: I_{\nu}} \text{ ,}
\end{equation}
où $I_{\nu}$ est la densité spectrale d'intensité, reliée à la densité spectrale d'amplitude par $I_{\nu}=E_{\nu}E^*_{\nu}$. 

L'intensité totale moyenne est obtenue de manière directe en moyennant les densités spectrales d'intensité:
\begin{equation}
\overline{I}=\int{\diff\nu \: \overline{I_{\nu}}} \text{ .}
\end{equation}

La variance est obtenue en calculant le moment d'ordre 2 de l'intensité totale, qui s'écrit:
\begin{align}
\overline{I^2} &= \overline{\int{\diff\nu\diff\nu' \: I_{\nu} I_{\nu'}}} \\
&= \int{\diff\nu\diff\nu' \: \overline{I_{\nu} I_{\nu'}}} \\
&=\int{\diff\nu\diff\nu' \: \left( \overline{E_{\nu} E^*_{\nu}} \: \overline{E_{\nu'} E^*_{\nu'}} + \overline{E_{\nu} E^*_{\nu'}} \: \overline{E_{\nu'} E^*_{\nu}} \right)} \\
&= \overline{I}^2 + \int{\diff\nu\diff\nu' \: \overline{E_{\nu}E^*_{\nu'}}\:\overline{E_{\nu'}E^*_{\nu}}} \text{ ,}
\end{align}
à l'aide du théorème de Wick appliqué aux densités spectrales d'amplitude. On fait ainsi apparaître la fonction de corrélation $\overline{E_{\nu}E^*_{\nu'}}$ des densités spectrales d'amplitude.

On détermine finalement que la variance de l'intensité du champ de \speckle\ est donnée par:
\begin{align}
\sigma_I^2 &= \overline{I^2} - \overline{I}^2 \\
&= \int{\diff\nu\diff\nu' \: \left|\overline{E_{\nu}E^*_{\nu'}} \right|^2} \text{ ,}
\end{align}
et est donc directement reliée à la fonction de corrélation à deux corps bichromatique identifiée précédemment.