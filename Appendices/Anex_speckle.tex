\chapter{Calculs de champs de tavelures}
\section{Amplitude rayonnée}
Notons $\mathbf{x} \equiv \left\lbrace x,y,d \right\rbrace$ et $\mathbf{x}_0 \equiv \left\lbrace x_0,y_0,0 \right\rbrace$
Pour calculer le champ rayonné au point $\mathbf{x}$, on utilise le principe de Huygens-Fresnel:
\begin{equation}
E(\mathbf{x})=\frac{1}{i \lambda} \int {\mathrm{d}\mathbf{x}_0 \: t(\mathbf{x}_0) E_0(\mathbf{x}_0) \frac{e^{ik \left| \mathbf{x}-\mathbf{x}_0 \right| }}{\left| \mathbf{x} - \mathbf{x}_0 \right| }}
\label{eq:Huygens-Fresnel}
\end{equation}
avec $t(\mathbf{x}_0)$ la transmittance complexe du diffuseur au point $\mathbf{x}_0$, $k=2\pi /\lambda$, et $t(\mathbf{x}_0)$ est une transmittance comportant l'effet du diffuseur et de la lentille.
Appliquons alors l'approximation paraxiale:
\begin{align}
\nonumber \left| \mathbf{x} - \mathbf{x}_0 \right| &= \sqrt{(x-x_0)^2+(y-y_0)^2+d^2} \\
\nonumber & = d \sqrt{1+\frac{(x-x_0)^2+(y-y_0)^2}{d^2}} \\
& \approx d+ \frac{(x-x_0)^2+(y-y_0)^2}{2d}
\label{eq:paraxial}
\end{align}
et reportons \ref{eq:paraxial} dans \ref{eq:Huygens-Fresnel}:
\begin{equation}
E(\mathbf{x})=\frac{e^{ikd}}{i \lambda d} \int{\mathrm{d} \mathbf{x}_0 \: t_{\mathrm{diff}}(\mathbf{x}_0) E_0(\mathbf{x}_0) \: e^{-ik \frac{x_0^2+y_0^2}{2f}} e^{ik \frac{(x-x_0)^2+(y-y_0)^2}{2d}}}
\end{equation}
avec $t_{\mathrm{diff}}(\mathbf{x}_0)$ la transmittance du diffuseur. Développons alors cette dernière expression:
\begin{equation}
E(x,y,d)=\frac{e^{i k \left( d + \frac{x^2 + y^2}{2 d}\right) }}{i \lambda d} \int{\mathrm{d} \mathbf{x}_0 \: t_{\mathrm{diff}}(\mathbf{x}_0) E_0(\mathbf{x}_0) \: e^{ik \frac{(x_0^2+y_0^2)}{2 d_{eff}}} e^{-ik\frac{(x x_0 + y y_0)}{d}}}
\end{equation}
avec $1/d_{eff}=1/d-1/f$.

\section{Fonction de corrélation}
\begin{equation}
\begin{split}
C _{\delta I} \left( x_1, x_2, \lambda _1, \lambda _2 \right) &= \left\langle \left( I \left( x_1, \lambda _1 \right) - \left\langle I \left( x_1, \lambda _1 \right) \right\rangle \right) \left( I \left( x_2, \lambda _2 \right) - \left\langle I \left( x_2, \lambda _2 \right) \right\rangle \right) \right\rangle \\
&= \left| C_E \left( x_1, x_2, \lambda _1, \lambda _2 \right) \right| ^2
\end{split}
\end{equation}
avec
\begin{equation}
C_E(x_1 ,x_2 ,y_1 ,y_2 ,d_1 ,d_2 ,\lambda _1,\lambda _2)= \left\langle E(x_1 ,y_1 ,d_1 ,\lambda _1) E^*(x_2 ,y_2 ,d_2 ,\lambda _2) \right\rangle
\end{equation}
donne:
\begin{equation}
C_E \propto \mathrm{TF} \left[ I_{eff}(x_0,y_0) \times G_{0eff}(x_0,y_0) \right] _{\left[ \frac{x_1 k_1}{d_1}-\frac{x_2 k_2}{d_2} \right]}
\end{equation}
avec
\begin{equation*}
I_{eff}(x_0)= I_0(x_0) \times e^{i \frac{x_0^2}{2} \left( \frac{k_1}{d_{eff1}} + \frac{k_2}{d_{eff2}}\right)}
\end{equation*}
\begin{equation*}
G_{0eff}(x_0)=\mathrm{TF}^{-1} \left( C_{diff}(\Delta x) \right)_{\left[ \frac{x_0}{2} \left( \frac{k_1}{d_{eff1}} + \frac{k_2}{d_{eff2}} \right) -\frac{x_1 k_1}{d_1} - \frac{x_2 k_2}{d_2}\right]}
\end{equation*}