\chapter{Calculs du temps de diffusion élastique}
\label{ch:anex_taus}

\section{Élements de calcul du temps de diffusion élastique à partir du développement de Born}
La fonction de Green moyennée est donnée par 
\begin{equation}
\widetilde{G}(E,\mathbf{k})=\frac{1}{E-E_{\mathrm{k}}-\overline{V}-\Sigma(E,\mathbf{k})}
\end{equation}


\begin{equation}
\Sigma=\Sigma^{(1)} + \Sigma^{(2)} + \dots
\end{equation}
avec $\Sigma^{(1)}=\overline{V G_0 V}$ et $\Sigma^{(2)}=\overline{V G_0 V G_0 V}$.

\begin{equation}
\overline{V((\mathbf{x}_{\mathrm{1}})V(\mathbf{x}_{\mathrm{2}})}=\VR^2 e^{-\frac{(\mathbf{x}_{\mathrm{2}}-\mathbf{x}_{\mathrm{1}})^2}{\sigma^2}}
\end{equation}

\begin{align}
\overline{V(\mathbf{x}_{\mathrm{1}})V(\mathbf{x}_{\mathrm{2}})V(\mathbf{x}_{\mathrm{3}})}&= 2\VR^3 e^{-\frac{(\mathbf{x}_{\mathrm{2}}-\mathbf{x}_{\mathrm{1}})^2}{2\sigma^2}}e^{-\frac{(\mathbf{x}_{\mathrm{3}}-\mathbf{x}_{\mathrm{2}})^2}{2\sigma^2}}e^{-\frac{(\mathbf{x}_{\mathrm{1}}-\mathbf{x}_{\mathrm{3}})^2}{2\sigma^2}} \\ 
&=2\VR^3 \gamma(\mathbf{x}_{\mathrm{2}}-\mathbf{x}_{\mathrm{1}})\gamma(\mathbf{x}_{\mathrm{3}}-\mathbf{x}_{\mathrm{2}})\gamma(\mathbf{x}_{\mathrm{1}}-\mathbf{x}_{\mathrm{3}})
\end{align}


\section{Calcul de la Self-Energy au premier ordre}

\subsection{Expression du premier ordre de la self-energy}
Partons de l'expression du premier terme de la self-energy et et évaluons-le entre deux états $\etat{\mathbf{k}}$ et $\etat{\mathbf{k}'}$:
\begin{align}
\left\langle \mathbf{k} | \Sigma^{(1)}(E) | \mathbf{k}' \right\rangle &= \left\langle \mathbf{k} | \overline{V G_0(E) V} | \mathbf{k}' \right\rangle \\
&= \int{ \diff \mathbf{x}_{\mathrm{1}} \diff \mathbf{x}_{\mathrm{2}} \: \left\langle \mathbf{k} | \mathbf{x}_{\mathrm{1}} \right\rangle \overline{V(\mathbf{x}_{\mathrm{1}}) \left\langle \mathbf{x}_{\mathrm{1}} | G_0(E) | \mathbf{x}_{\mathrm{2}}\right\rangle V(\mathbf{x}_{\mathrm{2}}) } \left\langle \mathbf{x}_{\mathrm{2}} |\mathbf{k}' \right\rangle} \\
&= \sum_{\mathbf{k}_{\mathrm{1}}}{\int{\diff\mathbf{x}_{\mathrm{1}} \diff\mathbf{x}_{\mathrm{2}} \: \left\langle \mathbf{k} |\mathbf{x}_{\mathrm{1}} \right\rangle \left\langle \mathbf{x}_{\mathrm{2}} | \mathbf{k}' \right\rangle \left\langle \mathbf{x}_{\mathrm{1}} | \mathbf{k}_{\mathrm{1}} \right\rangle \left\langle \mathbf{k}_{\mathrm{1}} | \mathbf{x}_{\mathrm{2}} \right\rangle G_0(E,\mathbf{k}_{\mathrm{1}})\overline{V(\mathbf{x}_{\mathrm{1}})V(\mathbf{x}_{\mathrm{2}})}}} \\
&= \lim\limits_{L\rightarrow\infty} \left( \sum_{\mathbf{k}_{\mathrm{1}}}{\int{\diff\mathbf{x}_{\mathrm{1}}\diff\mathbf{x}_{\mathrm{2}}\: C(\mathbf{x}_{\mathrm{2}}-\mathbf{x}_{\mathrm{1}}) G_0(E,\mathbf{k}_{\mathrm{1}}) \frac{1}{L^{2d}} e^{i \mathbf{k}_{\mathrm{1}}(\mathbf{x}_{\mathrm{1}}-\mathbf{x}_{\mathrm{2}})} e^{i \mathbf{x}_{\mathrm{2}} \mathbf{k}'} e^{-i \mathbf{x}_{\mathrm{1}} \mathbf{k}} }} \right) \\
&= \lim\limits_{L\rightarrow\infty} \left( \int{\diff\mathbf{k}_{\mathrm{1}} \: \left(\frac{L}{2\pi}\right)^d G_0(E,\mathbf{k}_{\mathrm{1}}) \int{\diff\Delta\mathbf{x} \: C(\Delta\mathbf{x}) \frac{e^{i \Delta\mathbf{x}(\mathbf{k}'-\mathbf{k}_{\mathrm{1}})}}{L^d}} \int{\diff\mathbf{x}_{\mathrm{1}} \: \frac{e^{i \mathbf{x}_{\mathrm{1}}(\mathbf{k}'-\mathbf{k})}}{L^d}} } \right) \\
&=\int{\frac{\diff\mathbf{k}_{\mathrm{1}}}{(2\pi)^d} \: G_0(E,\mathbf{k}_{\mathrm{1}}) \lim\limits_{L\rightarrow\infty}\left( \int{\diff\Delta\mathbf{x} \: C(\Delta\mathbf{x}) e^{i\Delta\mathbf{x}(\mathbf{k}'-\mathbf{k}_{\mathrm{1}})}} \int{\diff\mathbf{x}_{\mathrm{1}} \: \left\langle \mathbf{k} | \mathbf{x}_{\mathrm{1}}\right\rangle \left\langle \mathbf{x}_{\mathrm{1}} | \mathbf{k}' \right\rangle } \right)} \\
&= \left\langle \mathbf{k} | \mathbf{k}' \right\rangle \int{\frac{\diff\mathbf{k}_{\mathrm{1}}}{(2\pi)^d} \: G_0(E,\mathbf{k}_{\mathrm{1}}) \widetilde{C}(\mathbf{k}'-\mathbf{k}_{\mathrm{1}})} \text{ .}
\end{align}
On retrouve ainsi que la self-energy est diagonale dans l'espace des $\etat{\mathbf{k}}$, témoignant de l'invariance par translation du système, et que celle-ci est donnée par la convolution de la fonction de Green libre avec le spectre des fréquences spatiales du désordre.


\subsection{Calcul du temps de diffusion élastique}
Le temps de diffusion élastique est relié à la partie imaginaire de la self-energy:
\begin{align}
\mathrm{Im}[\Sigma^{(1)}(E,\mathbf{k})]&= \mathrm{Im} \left[ \int{\frac{\diff\mathbf{k}_{\mathrm{1}}}{4\pi^2}\: G_0(E,\mathbf{k}_{\mathrm{1}}) \widetilde{C}(\mathbf{k}-\mathbf{k}_{\mathrm{1}})}\right] \\ 
&= \int{\frac{\diff\mathbf{k}_{\mathrm{1}}}{4\pi^2}\: \mathrm{Im}[G_0(E,\mathbf{k}_{\mathrm{1}})] \widetilde{C}(\mathbf{k}-\mathbf{k}_{\mathrm{1}})} \\
&= -\pi \int{\frac{\diff\mathbf{k}_{\mathrm{1}}}{4\pi^2} \delta(E-E_{\mathrm{k}_1}) \widetilde{C}(\mathbf{k}-\mathbf{k}_{\mathrm{1}})} \\
&= - \pi \VR^2 \int{\frac{\diff\mathbf{k}_{\mathrm{1}}}{4\pi^2} \: \pi \sigma^2 e^{-(\mathbf{k}-\mathbf{k}_{\mathrm{1}})^2 \sigma^2/4} \delta(E-E_{\mathrm{k}_1})} \\
&=-\frac{\VR^2\sigma^2}{4} \int{\diff\mathbf{k}_{\mathrm{1}} \: e^{-k^2\sigma^2/4} e^{-k_{\mathrm{1}}^2\sigma^2/4} e^{\mathbf{k}\mathbf{k}_{\mathrm{1}}\sigma^2/4} \delta(E-E_{\mathrm{k}_1})}\\
&= -\frac{\VR^2\sigma^2}{4} \int{\diff k_{\mathrm{1}} k_{\mathrm{1}} \diff\theta_{\mathrm{1}} \: e^{-k^2\sigma^2/4} e^{-k_{\mathrm{1}}^2\sigma^2/4} e^{k k_{\mathrm{1}} \cos{\theta_{\mathrm{3}}}} \delta(E-E_{\mathrm{k}_1})} \\
&= -\frac{\pi}{2} \VR^2\sigma^2 \int{\diff k_{\mathrm{1}} k_{\mathrm{1}} \: e^{-k^2\sigma^2/4} e^{-k_{\mathrm{1}}^2\sigma^2/4} \mathrm{I_0}(k k_{\mathrm{1}}\sigma^2/2) \delta(E-E_{\mathrm{k}_1})} \text{ ,}
\end{align}
où $\mathrm{I_0}(x)$ est la fonction de Bessel modifiée d'ordre 0, donnée par
\begin{equation}
\int_{0}^{2\pi}{\diff\theta\:e^{x\cos\theta}}=2\pi \mathrm{I_0}(x) \quad \text{pour} \quad x>0 \text{ .}
\end{equation}

On fait maintenant l'approximation en couche $E=E_{\mathrm{k}}$, et on fait apparaître le temps de diffusion élastique
\begin{align}
\frac{\hb}{\taus}&=-2\mathrm{Im}[\Sigma^{(1)}(E_{\mathrm{k}},\mathbf{k})] \\
&= \pi \frac{\VR^2}{\ER} \int{\diff k_{\mathrm{1}}^2 \: e^{-k^2\sigma^2 /4} e^{-k_{\mathrm{1}}^2\sigma^2/4} \mathrm{I_0}(k k_{\mathrm{1}} \sigma^2 /2) \delta(k^2-k_{\mathrm{1}}^2)} \\
&= \pi \frac{\VR^2}{\ER} e^{-k^2\sigma^2/2} \mathrm{I_0}(k^2\sigma^2/2) \text{ .}
\end{align}
On retrouve ainsi l'expression de $\taus^{\mathrm{Born}}$ en deux dimensions annoncée dans le chapitre \ref{ch:TauS_PRL}.






\section{Calcul de la Self-Energy au second ordre}



\begin{equation}
\widetilde{G_0}(\mathbf{k})=\frac{1}{E-k^2/2}-i \pi \delta(E-k^2/2)
\end{equation}

\begin{align}
\gamma(\mathbf{x})=e^{-\frac{\mathrm{x}^2}{2\sigma^2}} \quad \rightarrow \quad \tilde{\gamma}(\mathbf{k}) &= (2\pi \sigma^2)^{d/2} e^{-\frac{k^2\sigma^2}{2}} \\
&= (2\pi \sigma^2)^{d/2} V(\mathbf{k})
\end{align}

\subsection{Calcul à l'aide la représentation en impulsion}
Commençons par évaluer le deuxième ordre de la self-energy dans l'espace des impulsions, entre les états $\etat{\mathbf{k}}$ et $\etat{\mathbf{k}'}$:
\begin{align}
\left\langle \mathbf{k} | \Sigma^{(2)}(E) | \mathbf{k}' \right\rangle &= \left\langle \mathbf{k} | \overline{V G_0(E) V G_0(E) V} | \mathbf{k}' \right\rangle \\
&=\int{\diff\mathbf{x}_{\mathrm{1}}\diff\mathbf{x}_{\mathrm{2}}\diff\mathbf{x}_{\mathrm{3}}\:}  \\
&\nonumber\qquad\times \langle\mathbf{k}|\mathbf{x}_{\mathrm{1}}\rangle \overline{V(\mathbf{x}_{\mathrm{1}}) \langle\mathbf{x}_{\mathrm{1}}|G_0(E)|\mathbf{x}_{\mathrm{2}}\rangle V(\mathbf{x}_{\mathrm{2}}) \langle\mathbf{x}_{\mathrm{2}}|G_0(E)|\mathbf{x}_{\mathrm{3}}\rangle V(\mathbf{x}_{\mathrm{3}})} \langle\mathbf{x}_{\mathrm{3}}|\mathbf{k}'\rangle \\
&= \sum_{\mathbf{k}_{\mathrm{1}},\mathbf{k}_{\mathrm{2}}}{ G_0(E,\mathbf{k}_{\mathrm{1}}) G_0(E,\mathbf{k}_{\mathrm{2}})} \int{\diff\mathbf{x}_{\mathrm{1}} \diff\mathbf{x}_{\mathrm{2}} \diff\mathbf{x}_{\mathrm{3}}} \\
&\nonumber\qquad\times \overline{V(\mathbf{x}_{\mathrm{1}})V(\mathbf{x}_{\mathrm{2}})V(\mathbf{x}_{\mathrm{3}})} e^{-i(\mathbf{k}-\mathbf{k}_{\mathrm{1}})\mathbf{x}_{\mathrm{1}}} e^{-i(\mathbf{k}_{\mathrm{1}}-\mathbf{k}_{\mathrm{2}})\mathbf{x}_{\mathrm{2}}} e^{-i(\mathbf{k}_{\mathrm{2}}-\mathbf{k}')\mathbf{x}_{\mathrm{3}}} \\
&=2 \VR^3 \sum_{\mathbf{k}_{\mathrm{1}},\mathbf{k}_{\mathrm{2}}}{ G_0(E,\mathbf{k}_{\mathrm{1}}) G_0(E,\mathbf{k}_{\mathrm{2}})} \int{\diff\mathbf{x}_{\mathrm{1}} \diff\mathbf{x}_{\mathrm{2}} \diff\mathbf{x}_{\mathrm{3}}} \\
&\nonumber\qquad\times \gamma(\mathbf{x}_{\mathrm{1}}-\mathbf{x}_{\mathrm{2}})\gamma(\mathbf{x}_{\mathrm{2}}-\mathbf{x}_{\mathrm{3}})\gamma(\mathbf{x}_{\mathrm{3}}-\mathbf{x}_{\mathrm{1}}) e^{-i(\mathbf{k}-\mathbf{k}_{\mathrm{1}})\mathbf{x}_{\mathrm{1}}} e^{-i(\mathbf{k}_{\mathrm{1}}-\mathbf{k}_{\mathrm{2}})\mathbf{x}_{\mathrm{2}}} e^{-i(\mathbf{k}_{\mathrm{2}}-\mathbf{k}')\mathbf{x}_{\mathrm{3}}} \\
&= 2 \VR^3 \sum_{\mathbf{k}_{\mathrm{1}},\mathbf{k}_{\mathrm{2}}}{G_0(E,\mathbf{k}_{\mathrm{1}}) G_0(E,\mathbf{k}_{\mathrm{2}})} \int{\diff\mathbf{x}_{\mathrm{1}}'\diff\mathbf{x}_{\mathrm{2}}'\diff\mathbf{x}_{\mathrm{3}}' \: \gamma(\mathbf{x}_{\mathrm{1}}')\gamma(\mathbf{x}_{\mathrm{2}}')\gamma(\mathbf{x}_{\mathrm{1}}'-\mathbf{x}_{\mathrm{2}}')}\\
&\nonumber \qquad\times e^{-i(\mathbf{k}-\mathbf{k}_{\mathrm{1}})\mathbf{x}_{\mathrm{1}}'} e^{-i(\mathbf{k}_{\mathrm{2}}'-\mathbf{k}')\mathbf{x}_{\mathrm{2}}'} e^{-i(\mathbf{k}-\mathbf{k}')\mathbf{x}_{\mathrm{3}}'} \\
&=\langle \mathbf{k} | \mathbf{k}'\rangle \times 2 \VR^3 \sum_{\mathbf{k}_{\mathrm{1}},\mathbf{k}_{\mathrm{2}}}{G_0(E,\mathbf{k}_{\mathrm{1}}) G_0(E,\mathbf{k}_{\mathrm{2}})} \int{\diff\mathbf{x}_{\mathrm{1}}'\diff\mathbf{x}_{\mathrm{2}}'} \\
&\nonumber \qquad\times \gamma(\mathbf{x}_{\mathrm{1}}')\gamma(\mathbf{x}_{\mathrm{2}}')\gamma(\mathbf{x}_{\mathrm{1}}'-\mathbf{x}_{\mathrm{2}}') e^{-i(\mathbf{k}-\mathbf{k}_{\mathrm{1}})\mathbf{x}_{\mathrm{1}}'} e^{-i(\mathbf{k}_{\mathrm{2}}'-\mathbf{k}')\mathbf{x}_{\mathrm{2}}'}
\end{align}
en réalisant le changement de variables $\lbrace \mathbf{x}_{\mathrm{1}},\mathbf{x}_{\mathrm{2}},\mathbf{x}_{\mathrm{3}}\rbrace \rightarrow \lbrace \mathbf{x}_{\mathrm{1}}'=\mathbf{x}_{\mathrm{1}}-\mathbf{x}_{\mathrm{2}}, \mathbf{x}_{\mathrm{2}}'=\mathbf{x}_{\mathrm{3}}-\mathbf{x}_{\mathrm{2}}, \mathbf{x}_{\mathrm{3}}'=\mathbf{x}_{\mathrm{2}}\rbrace$. On fait une nouvelle fois apparaître le fait que la self-energy est diagonale dans l'espace des $\etat{\mathbf{k}}$, traduisant l'invariance par translation du système. 

On peut alors donner une expression simple du deuxième ordre de la self-energy
\begin{align}
\Sigma^{(2)}(E,\mathbf{k})&= 2\VR^3 \sum_{\mathbf{k}_{\mathrm{1}},\mathbf{k}_{\mathrm{2}}}{G_0(E,\mathbf{k}_{\mathrm{1}}) G_0(E,\mathbf{k}_{\mathrm{2}})} \int{\diff\mathbf{x}_{\mathrm{1}}' \diff\mathbf{x}_{\mathrm{2}}' \: \gamma(\mathbf{x}_{\mathrm{1}}')\gamma(\mathbf{x}_{\mathrm{2}}')\gamma(\mathbf{x}_{\mathrm{1}}'-\mathbf{x}_{\mathrm{2}}')}\\
&\nonumber\qquad\times e^{-i(\mathbf{k}-\mathbf{k}_{\mathrm{1}})\mathbf{x}_{\mathrm{1}}'} e^{-i(\mathbf{k}_{\mathrm{2}}-\mathbf{k})\mathbf{x}_{\mathrm{2}}'} \\
&= \sum_{\mathbf{k}_{\mathrm{1}},\mathbf{k}_{\mathrm{2}}}{G_0(E,\mathbf{k}_{\mathrm{1}}) G_0(E,\mathbf{k}_{\mathrm{2}}) \widetilde{C}^{(3)}(\mathbf{k}-\mathbf{k}_{\mathrm{1}},\mathbf{k}-\mathbf{k}_{\mathrm{2}})} \text{ ,}
\end{align}
qui correspond à la formule mentionnée dans le chapitre \ref{ch:TauS_NJP}. Cette formule peut s'interpréter dans le cadre de la théorie des perturbations comme l'action multiple du potentiel sur l'état initial, propagé librement entre ces actions.



\subsection{Calcul à l'aide la représentation spatiale}
Il est possible d'obtenir une expression plus simple à évaluer numériquement en partant de la représentation spatiale du second ordre de la self-energy:
\begin{align}
\Sigma^{(2)}(\mathbf{x}_{\mathrm{1}},\mathbf{x}_{\mathrm{2}}) &= \int{\diff \mathbf{x} \: G_0(\mathbf{x}_{\mathrm{1}} - \mathbf{x}) G_0(\mathbf{x}-\mathbf{x}_{\mathrm{2}}) \overline{V(\mathbf{x}_{\mathrm{1}})V(\mathbf{x})V(\mathbf{x}_{\mathrm{2}})}} \\
&=2 \VR^3 \int{\diff\mathbf{x} \: G_0(\mathbf{x}_{\mathrm{1}} - \mathbf{x}) G_0(\mathbf{x}-\mathbf{x}_{\mathrm{2}}) \gamma(\mathbf{x}_{\mathrm{1}}-\mathbf{x})\gamma(\mathbf{x}-\mathbf{x}_{\mathrm{2}})\gamma(\mathbf{x}_{\mathrm{2}}-\mathbf{x}_{\mathrm{1}})} \\
&=2 \VR^3 \int{\diff\mathbf{x} \int{\frac{\diff\mathbf{k}_{\mathrm{1}}\diff\mathbf{k}_{\mathrm{2}}\diff\mathbf{k}_{\mathrm{3}}\diff\mathbf{k}_{\mathrm{4}}\diff\mathbf{k}_{\mathrm{5}}}{(2\pi)^{5d}}\: \widetilde{G_0}(\mathbf{k}_{\mathrm{1}}) \widetilde{G_0}(\mathbf{k}_{\mathrm{2}}) \tilde{\gamma}(\mathbf{k}_{\mathrm{3}})\tilde{\gamma}(\mathbf{k}_{\mathrm{4}})\tilde{\gamma}(\mathbf{k}_{\mathrm{5}}) }} \\
&\nonumber \qquad\times e^{-i\mathbf{k}_{\mathrm{1}}(\mathbf{x}_{\mathrm{1}}-\mathbf{x})} e^{-i\mathbf{k}_{\mathrm{2}}(\mathbf{x}-\mathbf{x}_{\mathrm{2}})} e^{-i\mathbf{k}_{\mathrm{3}}(\mathbf{x}_{\mathrm{1}}-\mathbf{x})} e^{-i\mathbf{k}_{\mathrm{4}}(\mathbf{x}-\mathbf{x}_{\mathrm{2}})} e^{-i\mathbf{k}_{\mathrm{5}}(\mathbf{x}_{\mathrm{2}}-\mathbf{x}_{\mathrm{1}})} \\
&= \frac{2\VR^2}{(2\pi)^{5d}} \int{\diff\mathbf{k}_{\mathrm{1}}\diff\mathbf{k}_{\mathrm{2}}\diff\mathbf{k}_{\mathrm{3}}\diff\mathbf{k}_{\mathrm{4}}\diff\mathbf{k}_{\mathrm{5}}\: \widetilde{G_0}(\mathbf{k}_{\mathrm{1}}) \widetilde{G_0}(\mathbf{k}_{\mathrm{2}}) \tilde{\gamma}(\mathbf{k}_{\mathrm{3}})\tilde{\gamma}(\mathbf{k}_{\mathrm{4}})\tilde{\gamma}(\mathbf{k}_{\mathrm{5}})}\\
&\nonumber \qquad \times e^{-i\mathbf{x}_{\mathrm{1}} (\mathbf{k}_{\mathrm{1}}+\mathbf{k}_{\mathrm{3}}-\mathbf{k}_{\mathrm{5}})} e^{-i\mathbf{x}_{\mathrm{2}}(\mathbf{k}_{\mathrm{5}}-\mathbf{k}_{\mathrm{4}}-\mathbf{k}_{\mathrm{2}})} \int{\diff\mathbf{x}\: e^{-i\mathbf{x}(\mathbf{k}_{\mathrm{2}}-\mathbf{k}_{\mathrm{1}}-\mathbf{k}_{\mathrm{3}}+\mathbf{k}_{\mathrm{4}})}}\\
&=\frac{2\VR^3}{(2\pi)^{4d}} \int{\diff\mathbf{k}_{\mathrm{1}}\diff\mathbf{k}_{\mathrm{2}}\diff\mathbf{k}_{\mathrm{3}}\diff\mathbf{k}_{\mathrm{4}}\diff\mathbf{k}_{\mathrm{5}}\: \widetilde{G_0}(\mathbf{k}_{\mathrm{1}}) \widetilde{G_0}(\mathbf{k}_{\mathrm{2}}) \tilde{\gamma}(\mathbf{k}_{\mathrm{3}})\tilde{\gamma}(\mathbf{k}_{\mathrm{4}})\tilde{\gamma}(\mathbf{k}_{\mathrm{5}})}\\
&\nonumber \qquad \times e^{-i\mathbf{x}_{\mathrm{1}} (\mathbf{k}_{\mathrm{1}}+\mathbf{k}_{\mathrm{3}}-\mathbf{k}_{\mathrm{5}})} e^{-i\mathbf{x}_{\mathrm{2}}(\mathbf{k}_{\mathrm{5}}-\mathbf{k}_{\mathrm{4}}-\mathbf{k}_{\mathrm{2}})} \delta(\mathbf{k}_{\mathrm{2}} +\mathbf{k}_{\mathrm{4}} -\mathbf{k}_{\mathrm{1}} -\mathbf{k}_{\mathrm{3}}) \\
&= \frac{2\VR^3}{(2\pi)^{4d}} \int{\diff\mathbf{k}_{\mathrm{2}}\diff\mathbf{k}_{\mathrm{3}}\diff\mathbf{k}_{\mathrm{4}}\diff\mathbf{k}_{\mathrm{5}}\: \widetilde{G_0}(\mathbf{k}_{\mathrm{2}}+\mathbf{k}_{\mathrm{4}}-\mathbf{k}_{\mathrm{3}}) \widetilde{G_0}(\mathbf{k}_{\mathrm{2}})\tilde{\gamma}(\mathbf{k}_{\mathrm{3}})\tilde{\gamma}(\mathbf{k}_{\mathrm{4}})\tilde{\gamma}(\mathbf{k}_{\mathrm{5}})} \\
&\nonumber \qquad \times e^{-i(\mathbf{k}_{\mathrm{2}}+\mathbf{k}_{\mathrm{4}}-\mathbf{k}_{\mathrm{5}})(\mathbf{x}_{\mathrm{1}}-\mathbf{x}_{\mathrm{2}})}\\
&=\Sigma^{(2)}(\mathbf{x}_{\mathrm{1}}-\mathbf{x}_{\mathrm{2}}) \text{ .}
\end{align}
Encore une fois, on retrouve l'invariance par translation.

La self-energy dans l'espace des impulsions est alors obtenue par transformée de Fourier: 
\begin{align}
\Sigma^{(2)}(\mathbf{k})&=\int{\diff\mathbf{x} \: e^{i \mathbf{k}\mathbf{x}} \Sigma^{(2)}(\mathbf{x})} \\
&= \frac{2\VR^3}{(2\pi)^{4d}} \int{\diff\mathbf{x} \: \int{\diff\mathbf{k}_{\mathrm{2}}\diff\mathbf{k}_{\mathrm{3}}\diff\mathbf{k}_{\mathrm{4}}\diff\mathbf{k}_{\mathrm{5}}\: \widetilde{G_0}(\mathbf{k}_{\mathrm{2}}+\mathbf{k}_{\mathrm{4}}-\mathbf{k}_{\mathrm{3}}) \widetilde{G_0}(\mathbf{k}_{\mathrm{2}})\tilde{\gamma}(\mathbf{k}_{\mathrm{3}})\tilde{\gamma}(\mathbf{k}_{\mathrm{4}})\tilde{\gamma}(\mathbf{k}_{\mathrm{5}})}} \\
&\nonumber \qquad \times e^{-i \mathbf{x} (\mathbf{k}_{\mathrm{2}} + \mathbf{k}_{\mathrm{4}} - \mathbf{k}_{\mathrm{5}} - \mathbf{k})} \\
&= \frac{2\VR^3}{(2\pi)^{3d}} \int{\diff\mathbf{k}_{\mathrm{2}}\diff\mathbf{k}_{\mathrm{3}}\diff\mathbf{k}_{\mathrm{4}}\diff\mathbf{k}_{\mathrm{5}}\: \widetilde{G_0}(\mathbf{k}_{\mathrm{2}}+\mathbf{k}_{\mathrm{4}}-\mathbf{k}_{\mathrm{3}}) \widetilde{G_0}(\mathbf{k}_{\mathrm{2}})\tilde{\gamma}(\mathbf{k}_{\mathrm{3}})\tilde{\gamma}(\mathbf{k}_{\mathrm{4}})\tilde{\gamma}(\mathbf{k}_{\mathrm{5}})}\\
&\nonumber \qquad \times\delta(\mathbf{k}_{\mathrm{2}} + \mathbf{k}_{\mathrm{4}} - \mathbf{k}_{\mathrm{5}} - \mathbf{k}) \\
&= \frac{2\VR^3}{(2\pi)^{3d}} \int{\diff\mathbf{k}_{\mathrm{3}} \diff\mathbf{k}_{\mathrm{4}}\diff\mathbf{k}_{\mathrm{5}} \: \widetilde{G_0}(\mathbf{k}+\mathbf{k}_{\mathrm{5}}-\mathbf{k}_{\mathrm{3}}) \widetilde{G_0}(\mathbf{k}+\mathbf{k}_{\mathrm{5}}-\mathbf{k}_{\mathrm{4}})\tilde{\gamma}(\mathbf{k}_{\mathrm{3}})\tilde{\gamma}(\mathbf{k}_{\mathrm{4}})\tilde{\gamma}(\mathbf{k}_{\mathrm{5}})} 
\end{align}

On fait le changement de variables $\mathbf{k}_{\mathrm{3}}'=\mathbf{k}+\mathbf{k}_{\mathrm{5}}-\mathbf{k}_{\mathrm{3}}$ et $\mathbf{k}_{\mathrm{4}}'=\mathbf{k}+\mathbf{k}_{\mathrm{5}}-\mathbf{k}_{\mathrm{4}}$:
\begin{align}
\Sigma^{(2)}(\mathbf{k})&= \frac{2\VR^3}{(2\pi)^{3d}} \int{\diff\mathbf{k}_{\mathrm{5}} \: \left( \int{\diff\mathbf{k}_{\mathrm{3}}'\: \widetilde{G_0}(\mathbf{k}_{\mathrm{3}}')\tilde{\gamma}(\mathbf{k}+\mathbf{k}_{\mathrm{5}}-\mathbf{k}_{\mathrm{3}}')} \right) \left( \int{\diff\mathbf{k}_{\mathrm{4}}'\: \widetilde{G_0}(\mathbf{k}_{\mathrm{4}}') \tilde{\gamma}(\mathbf{k}+\mathbf{k}_{\mathrm{5}}-\mathbf{k}_{\mathrm{4}}')}\right) \tilde{\gamma}(\mathbf{k}_{\mathrm{5}}) } \\
&= \frac{2\VR^3}{(2\pi)^{3d}} \int{\diff\mathbf{k}_{\mathrm{5}}' \: \left( \int{\diff\mathbf{k}_{\mathrm{3}}'\: \widetilde{G_0}(\mathbf{k}_{\mathrm{3}}')\tilde{\gamma}(\mathbf{k}-\mathbf{k}_{\mathrm{5}}'-\mathbf{k}_{\mathrm{3}}')} \right) \left( \int{\diff\mathbf{k}_{\mathrm{4}}'\: \widetilde{G_0}(\mathbf{k}_{\mathrm{4}}') \tilde{\gamma}(\mathbf{k}-\mathbf{k}_{\mathrm{5}}'-\mathbf{k}_{\mathrm{4}}')}\right) \tilde{\gamma}(\mathbf{k}_{\mathrm{5}}') }
\end{align}
en faisant le changement de variables $\mathbf{k}_{\mathrm{5}}'=-\mathbf{k}_{\mathrm{5}}$ et $\tilde{\gamma}(\mathbf{k}) $ est une fonction paire.

On fait le calcul pour $d=2$.
\begin{align}
\mathrm{Im}\left( \Sigma^{(2)}(\mathbf{k})\right)&= \frac{2\VR^3 (2\pi\sigma^2)^3}{(2\pi)^6} \\
&\nonumber \qquad\times\int{\diff\mathbf{k}_{\mathrm{5}}' \: \left( \int{\diff\mathbf{k}_{\mathrm{3}}' \: \frac{V(\mathbf{k}-\mathbf{k}_{\mathrm{5}}'-\mathbf{k}_{\mathrm{3}}')}{E-k_{\mathrm{3}}'^2/2}} (-\pi) \int{\diff\mathbf{k}_{\mathrm{4}}' \: \delta(E-k_{\mathrm{4}}'^2/2) \: V(\mathbf{k}-\mathbf{k}_{\mathrm{5}}'-\mathbf{k}_{\mathbf{4}}')} \right.} \\
&\nonumber \qquad \left.-\pi\int{\diff\mathbf{k}_{\mathrm{3}}' \: \delta(E-k_{\mathrm{3}}'^2/2) V(\mathbf{k}-\mathbf{k}_{\mathrm{5}}'-\mathbf{k}_{\mathrm{3}}')} \int{\diff\mathbf{k}_{\mathrm{4}}' \: \frac{V(\mathbf{k}-\mathbf{k}_{\mathrm{5}}'-\mathbf{k}_{\mathrm{4}}')}{E-k_{\mathrm{4}}'^2/2}}\right) V(\mathbf{k}_{\mathrm{5}}') \\
&= \frac{2\VR^3 (2\pi\sigma^2)^3}{(2\pi)^6} \int{\diff\mathbf{k}_{\mathrm{5}}' \: (-2\pi) V(\mathbf{k}_{\mathrm{5}}')} \\
&\nonumber\qquad \times\underbrace{\left(\int{\diff\mathbf{k}_{\mathrm{3}}' \:\frac{V(\mathbf{k}-\mathbf{k}_{\mathrm{5}}'-\mathbf{k}_{\mathrm{3}}')}{E-k_{\mathrm{3}}'^2/2}}\right)}_{(a)} \times \underbrace{\left(\int{\diff\mathbf{k}_{\mathrm{4}}'\:\delta(E-k_{\mathrm{4}}'^2/2) V(\mathbf{k}-\mathbf{k}_{\mathrm{5}}'-\mathbf{k}_{\mathrm{4}}')}\right)}_{(b)}
\end{align}

\begin{align}
(a)&= \int{\diff\mathbf{k}_{\mathrm{2}}' \: \frac{V(\mathbf{k}-\mathbf{k}_{\mathrm{5}}'-\mathbf{k}_{\mathrm{3}}')}{E-k_{\mathrm{3}}'^2/2}} \\
&= \int{\diff\mathbf{k}_{\mathrm{3}}' \: \frac{1}{E-k_{\mathrm{3}}'^2/2} e^{-(\mathbf{k}-\mathbf{k}_{\mathrm{5}}'-\mathbf{k}_{\mathrm{3}}')^2 \sigma^2/2}} \\
&= \int_{\R^+}{\diff k_{\mathrm{3}}' k_{\mathrm{3}}' \int_0^{\pi}{\diff\theta_{\mathrm{3}}'\: \frac{1}{E-k_{\mathrm{3}}'^2/2}e^{-\sigma^2/2\times (k_{\mathrm{3}}'^2+|\mathbf{k}-\mathbf{k}_{\mathrm{5}}'|^2-2k_{\mathrm{3}}' |\mathbf{k}-\mathbf{k}_{\mathrm{5}}'| \cos{\theta_{\mathrm{3}}'})}}}
\end{align}
et en utilisant $\int_0^{2\pi}{\diff\theta\:\exp{(x\cos{\theta})}} =2\pi \mathrm{I_0}(x)$ pour $x>0$.

\begin{align}
(a)&=\int_{\R^+}{\diff k_{\mathrm{3}}' k_{\mathrm{3}}' e^{-|\mathbf{k}-\mathbf{k}_{\mathrm{5}}'|^2 \sigma^2/2} \frac{e^{-k_{\mathrm{3}}'^2\sigma^2/2}}{E-k_{\mathrm{3}}'^2/2} \: 2\pi \mathrm{I_0}(k_{\mathrm{3}}' |\mathbf{k}-\mathbf{k}_{\mathrm{5}}'|\sigma^2)} \\
&= 2\pi e^{-\frac{|\mathbf{k}-\mathbf{k}_{\mathrm{5}}'|^2\sigma^2}{2}} \: \int_{\R^+}{\diff\epsilon'\:\frac{e^{-\epsilon'}}{\epsilon-\epsilon'}\mathrm{I_0}(|\mathbf{k}-\mathbf{k}_{\mathrm{5}}'|\sigma\sqrt{2\epsilon'})}
\end{align}
en faisant apparaître les énergies adimensionnées $\epsilon=E/\ER=E\sigma^2$ en unités naturelles. l'énergie cinétique s'écrit $\epsilon'=E_{\mathrm{k}}/\ER=k_{\mathrm{3}}'^2\sigma^2/2$.

\begin{align}
(b)&=\int{diff\mathbf{k}_{\mathrm{4}}' \: \delta(E-k_{\mathrm{4}}'^2/2) V(\mathbf{k}-\mathbf{k}_{\mathrm{5}}' -\mathbf{k}_{\mathrm{4}}')} \\
&= \int{\diff\mathbf{k}_{\mathrm{4}}' \: \delta(E-k_{\mathrm{4}}'^2/2) e^{-(\mathbf{k}-\mathbf{k}_{\mathrm{5}}'-\mathbf{k}_{\mathrm{4}}')\sigma^2/2}}\\
&=\int_{\R^+}{\diff k_{\mathrm{4}}' k_{\mathrm{4}}' \: \int_0^{2\pi}{\diff\theta_{\mathrm{4}}' \: \delta(E-k_{\mathrm{4}}'^2/2) e^{-|\mathbf{k}-\mathbf{k}_{\mathrm{5}}'|^2\sigma^2/2} e^{-\sigma^2 k_{\mathrm{4}}'^2/2} e^{|\mathbf{k}-\mathbf{k}_{\mathrm{5}}'| k_{\mathrm{4}}' \cos{\theta_{\mathrm{4}}'} } }} \\
&=2\pi e^{-\frac{|\mathbf{k}-\mathbf{k}_{\mathrm{5}}'|^2\sigma^2}{2}} \int_{\R^+}{\diff k_{\mathrm{4}}' k_{\mathrm{4}}' \:\delta(E- k_{\mathrm{4}}'^2/2) e^{-\sigma^2 k_{\mathrm{4}}'^2/2} \: \mathrm{I_0}(|\mathbf{k}-\mathbf{k}_{\mathrm{5}}'| \sigma^2 k_{\mathrm{4}}')}\\
&=2\pi e^{-\frac{|\mathbf{k}-\mathbf{k}_{\mathrm{5}}'|^2\sigma^2}{2}} \int_{\R^+}{\diff\epsilon' \:\delta(\epsilon-\epsilon') e^{-\epsilon'} \: \mathrm{I_0}(|\mathbf{k}-\mathbf{k}_{\mathrm{5}}'| \sigma \sqrt{2\epsilon'})}\\
&=2\pi e^{-\frac{|\mathbf{k}-\mathbf{k}_{\mathrm{5}}'|^2\sigma^2}{2}} e^{-\epsilon} \: \mathrm{I_0}(|\mathbf{k}-\mathbf{k}_{\mathrm{5}}'| \sigma \sqrt{2\epsilon})
\end{align}
avec $\epsilon'=k_{\mathrm{4}}'^2\sigma^2/2$.


\begin{align}
\mathrm{Im}\left( \Sigma^{(2)}(\mathbf{k})\right)&= \frac{2 \VR^3 (2\pi\sigma^2)^3)}{(2\pi)^6} \int{\diff\mathbf{k}_{\mathrm{5}}' \: (-2\pi) 2\pi e^{-|\mathbf{k}-\mathbf{k}_{\mathrm{5}}'|^2\sigma^2/2} \int_{\R^+}{\diff\epsilon' \: \frac{e^{-\epsilon'}}{\epsilon-\epsilon'} \mathrm{I_0}(|\mathbf{k}-\mathbf{k}_{\mathrm{5}}'|\sigma\sqrt{2\epsilon'})}}\\
&\nonumber \qquad \times 2\pi e^{-|\mathbf{k}-\mathbf{k}_{\mathrm{5}}'|^2\sigma^2/2} e^{-\epsilon} \mathrm{I_0}(|\mathbf{k}-\mathbf{k}_{\mathrm{5}}'|\sigma \sqrt{2\epsilon}) e^{-k_{\mathrm{5}}'^2\sigma^2/2} \\
 &= -2\VR^3\sigma^6 \int{\diff\mathbf{k}_{\mathrm{5}}' \: e^{-\epsilon} e^{-|\mathbf{k}-\mathbf{k}_{\mathrm{5}}'|^2 \sigma^2} e^{-k_{\mathrm{5}}'^2 \sigma^2/2} \mathrm{I_0}(|\mathbf{k}-\mathbf{k}_{\mathrm{5}}'|\sigma \sqrt{2}\epsilon)} \\
 &\nonumber \qquad \times \int_{\R^+}{\diff \epsilon' \: \frac{e^{-\epsilon'}}{\epsilon-\epsilon'} \mathrm{I_0}(|\mathbf{k}-\mathbf{k}_{\mathrm{5}}'|\sigma \sqrt{2\epsilon'})}
\end{align}



On fait le changement de variable $\mathbf{k}_{\mathrm{5}}''=\mathbf{k}-\mathbf{k}_{\mathrm{5}}'$:
\begin{align}
\mathrm{Im}\left( \Sigma^{(2)}(\mathbf{k})\right)&= -2\VR^3\sigma^6 \int{\diff\mathbf{k}_{\mathrm{5}}'' \: e^{-\epsilon} e^{-k_{\mathrm{5}}''^2\sigma^2} e^{|\mathbf{k}-\mathbf{k}_{\mathrm{5}}''|^2\sigma^2/2} \mathrm{I_0}(k_{\mathrm{5}}''\sigma \sqrt{2\epsilon})} \\
&\nonumber\qquad \times \int_{\R^+}{\diff\epsilon' \: \frac{e^{-\epsilon'}}{\epsilon-\epsilon'} \mathrm{I_0}(k_{\mathrm{5}}'' \sigma \sqrt{2\epsilon'})} \\
&=-4 \pi \VR^3\sigma^6 e^{-\epsilon} e^{-k^2\sigma^2/2} \int_{\R^+}{\diff k_{\mathrm{5}}'' \: k_{\mathrm{5}}'' e^{-\frac{3}{2}k_{\mathrm{5}}''^2 \sigma^2} \mathrm{I_0}(k_{\mathrm{5}}'' \sigma \sqrt{2\epsilon})} \\
&\nonumber \qquad \times \int_0^{2\pi}{\diff\theta_{\mathrm{5}}'' \: e^{\sigma^2 k k_{\mathrm{5}}'' \cos\theta_{\mathrm{5}}''} \int_{\R^+}{\diff \epsilon' \: \frac{e^{-\epsilon'}}{\epsilon-\epsilon'} \mathrm{I_0}(k_{\mathrm{5}}'' \sigma \sqrt{2\epsilon'})}} \\
&=-4 \pi \VR^3\sigma^6 e^{-\epsilon} e^{-k^2\sigma^2/2} \\
&\nonumber \qquad \times\int_{\R^+}{\diff k_{\mathrm{5}}'' \: k_{\mathrm{5}}'' e^{-\frac{3}{2}k_{\mathrm{5}}''^2 \sigma^2} \mathrm{I_0}(k_{\mathrm{5}}'' \sigma \sqrt{2\epsilon})^2 \int_{\R^+}{\diff \epsilon' \: \frac{e^{-\epsilon'}}{\epsilon-\epsilon'} \mathrm{I_0}(k_{\mathrm{5}}'' \sigma \sqrt{2\epsilon'})}} \\
&= -4 \pi \frac{\VR^3}{\ER^2} e^{-k^2\sigma^2} \int_{\R^+}{\diff k' \: k' e^{-\frac{3}{2}k'^2} \mathrm{I_0}(k' k \sigma)^2 \int_{\R^+}{\diff \epsilon' \: \frac{e^{-\epsilon'}}{k^2\sigma^2/2-\epsilon'} \mathrm{I_0}(k' \sqrt{2\epsilon'})}}
\end{align}
en posant $k'=k_{\mathrm{5}}''\sigma$ et en faisant l'approximation en couche $E=\hb^2k^2/2m \rightarrow \epsilon=k^2\sigma^2/2$. Cette dernière expression est celle que nous avons évalué numériquement pour déterminer les corrections au second ordre de $\taus^{\mathrm{sf}}$ dans le chapitre \ref{ch:TauS_NJP}.